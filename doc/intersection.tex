\chapter{Intersection of bit string representations}
\label{ch:intersectionofbitstringrepresentations}
\minitoc
%-------------------------------------------------
Now we look at the properties of an intersection of several bit strings.
%--------------------------------------------------
\section{Bit string addition}
\label{s:bitstringaddition}
%------------------------------------------------
Aussume, two bit strings are given. The intersection of its representative times tables is a \textbf{bitwise} addition of two numbers.

\begin{table}[H]
\centering
\caption{Bit string intersection for the case $x_{1} = 1$ and $x_{2} = 2$.}
\tiny
\begin{tabular}{cccccccccccccccccccccc||c}
				\cellcolor{yellow} 22 & 21 & 20 & 19 & 18 & 17 & 16 & 15 & 14 & 13 & 12 & 11 & 10 & 9 & 8 & \cellcolor{yellow} 7 & 6 & 5 & 4 & 3 & 2 & 1 & \\
\hline				21 & 20 & 19 & 18 & 17 & 16 & 15 & 14 & 13 & 12 & 11 & 10 & 9  & 8 & 7 & 6 & 5 & 4 & 3 & 2 & 1 & 0 & Index \\
\hline\hline \rowcolor{green}   \cellcolor{cyan} 1 &  0 &  0 &  \cellcolor{red} 1 &  0 &  \cellcolor{red} 0 &  \cellcolor{red} 1 &  0 &  0 &  \cellcolor{red} 1 &  \cellcolor{red} 0 &  0 &  \cellcolor{red} 1 & 0 & 0 & \cellcolor{cyan} 1 & 0 & 0 & \cellcolor{red} 1 & 0 & 0 & 0 & $x_{1} = 1$ \\
	     \rowcolor{green}   \cellcolor{cyan} 1 &  0 &  0 &  \cellcolor{red} 0 &  0 &  \cellcolor{red} 1 &  \cellcolor{red} 0 &  0 &  0 &  \cellcolor{red} 0 &  \cellcolor{red} 1 &  0 &  \cellcolor{red} 0 & 0 & 0 & \cellcolor{cyan} 1 & 0 & 0 & \cellcolor{red} 0 & 0 & 0 & 0 & $x_{2} = 2$ \\
\hline	     \rowcolor{green}	 \cellcolor{cyan} 1 &  0 &  0 &  \cellcolor{red} 1 &  0 &  \cellcolor{red} 1 &  \cellcolor{red} 1 &  0 &  0 &  \cellcolor{red} 1 &  \cellcolor{red} 1 &  0 &  \cellcolor{red} 1 & 0 & 0 & \cellcolor{cyan} 1 & 0 & 0 & \cellcolor{red} 1 & 0 & 0 & 0 & sum \\
\end{tabular}
\label{tab:intersectionx1_1+x2_2}
\end{table}

Neat. But bitwise addition is not really nice to handle, if we want a closed solution for our problem!

\vspace{0.3cm}
But have a look at the cases in which we have the addition $1 \ \& \ 1$ (cyan), which belongs to the numbers 7 and 22 (yellow).\\
In the next sections we will look for our valid lower and upper bounds, ranges and bits of our recursion and the belonging $x_{i,j}$ values. We will see that, in practice, we will never have the case of the addition of 1 and 1. This is great, since it means that we will be able to do a normal addition, instead of a bitwise addition!

\vspace{0.3cm}
Let's have a look, to understand, how we can reach this!
%--------------------------------------------------
\section{Intersection bounds, ranges and bits}
\label{s:intersectionboundsandranges}
%------------------------------------------------
What are our recursion bounds, ranges and bits?
%--------------------------------------------------
\subsection{Lower bits}
\label{ss:lowerbits}
%------------------------------------------------
Ok. If we look at equation (\ref{eq:xij_iconst}), we can see that it is symmetric for $x_{i}$ and $x_{j}$. Which means

\begin{alignat}{3}
	x_{i,j}\left(x_{i},x_{j}\right) :&= \left(2x_{i} + 1\right)x_{j} + x_{i} \notag \\
					& = \left(2x_{j} + 1\right)x_{i} + x_{j} \notag \\
					& = x_{i,j}\left(x_{j},x_{i}\right)
\label{eq:sym}\end{alignat}















%------------------------------------------------
% why we never have 22 => no intersection for x values which represented divisible numbers
%------------------------------------------------
