\chapter{Odd-Divisible Numbers}
\label{ch:odddivisiblenumbers}
\minitoc
%--------------------------------------------------
At first, for the description of prime numbers, we have to look at the set of divisible numbers. Since, apart from $2$, all prime numbers are odd, we will only analyze this numbers. In the whole paper, we will ignore the prime number $2$, because we will see, this makes a lot easier.
%--------------------------------------------------
\section{Basic description: Odd-Numbers}
\label{s:basicdescriptionoddnumbers}
%--------------------------------------------------
Be given the set of all odd natural numbers $y \in \mathbb{N}_{> 1}$ through
\begin{equation}
y_{i}\left(x_{i}\right) := 2x_{i} + 1,
\label{eq:oddnatnumbers}
\end{equation}

with $x,i \in \mathbb{N}$. If we expand the definition set of $x$ to $\mathbb{Z}$, we also know
\begin{alignat}{3}
	y\left(0\right) &= 1 \\
	\mathrm{and} \quad y(-x) &= 2\left(-x\right) + 1 \notag \\
	&= -\left(2x - 1\right) \notag \\
	&= -\left(2\left(x-1\right) + 1\right) \notag \\ 
	&= -y\left(x-1\right).
\label{eq:xexpandtoZ}
\end{alignat}

Later, we will see that this properties can be very useful.
%--------------------------------------------------
\section{Basic description: Odd-Divisible Numbers}
\label{s:basicdescriptionodddivisiblenumbers}
%--------------------------------------------------
Next, we look at all odd-divisible numbers. We know, they can't have a factor which is a multiple of $2$. Hence, we get an equation which describes all odd-divisible numbers by
\begin{alignat}{3}
	y_{i,j}\left(x_{i},x_{j}\right) &= y_{i}\left(x_{i}\right) \cdot y_{j}\left(x_{j}\right) \notag \\
	&= \left(2x_{i} + 1\right)\left(2x_{j} + 1\right) \notag \\
	&= 2^{2} x_{i} x_{j} + 2x_{i} + 2x_{j} + 1 \notag \\
	&= 2\left(\underbrace{2x_{i}x_{j} + x_{i} + x_{j}}_{=:x_{i,j}}\right) + 1 \notag \\
	&= y_{i,j}\left(x_{i,j}\right).
\label{eq:odddivnumbers}
\end{alignat}

If we expand again our sets to $\mathbb{Z}$, we receive additional cases. At first, assume at one factor is $y\left(0\right) = 1$. We see directly
\begin{alignat}{3}
	y_{0,j}\left(0,x_{j}\right) &= y_{0}\left(0\right) \cdot y_{j}\left(x_{j}\right) \notag \\
	&= 1 \cdot \left(2x_{j} + 1\right) \notag \\
	&= 2x_{j} + 1 \notag \\
	&= y_{j}\left(x_{j}\right) \\
	\mathrm{respectively} \quad y_{i,0}\left(x_{i},0\right) &= y_{i}\left(x_{i}\right).
\label{eq:onefactor1}
\end{alignat}

Next, assume we have one factor with $y\left(-x\right)$.
\begin{alignat}{3}
	y_{i,j}\left(-x_{i},x_{j}\right) & = y_{i}\left(-x_{i}\right) \cdot y_{j}\left(x_{j}\right) \notag \\
	&= \left(2\left(-x_{i}\right) + 1\right)\left(2x_{j} + 1\right) \notag \\
	&= -2^{2}x_{i}x_{j} - 2x_{i} + 2x_{j} + 1 \notag \\
	&= -\left(2\left(2x_{i}x_{j} + x_{i} - x_{j} - 1\right) + 1\right) \notag \\
	&= -\left(2\left(2x_{i}x_{j} + x_{i} - 2x_{j} + x_{j} -1\right) + 1\right) \notag \\
	&= -\left(2\left(2\left(x_{i} - 1\right)x_{j} + \left(x_{i} - 1\right) + x_{j}\right) + 1\right) \notag \\
	&= -y_{i}\left(x_{i} - 1\right) \cdot y_{j}\left(x_{j}\right) \\
	\mathrm{respectively} \quad y_{i,j}\left(x_{i},-x_{j}\right) &= -y_{i}\left(x_{i}\right) \cdot y_{j}\left(x_{j} - 1\right)
\label{eq:onefactorminus}
\end{alignat}

In the case of two negative factors, we have
\begin{alignat}{3}
	y_{i,j}\left(-x_{i},-x_{j}\right) &= y_{i}\left(-x_{i}\right) \cdot y_{j}\left(-x_{j}\right) \notag \\
	&= \left(2\left(-x_{i}\right) + 1\right)\left(2\left(-x_{j}\right) + 1\right) \notag \\
	&= \left(2x_{i} - 1\right)\left(2x_{j} - 1\right) \notag \\
	&= 2^{2}x_{i}x_{j} - 2x_{i} - 2x_{j} + 1 \notag \\
	&= 2\left(2x_{i}x_{j} - x_{i} - x_{j}\right) + 1 \notag \\
	&= \left(2x_{i} - 2 + 1\right)\left(2x_{j} - 2 + 1\right) \notag \\
	&= \left(2\left(x_{i} - 1\right) + 1\right)\left(2\left(x_{j} - 1\right) + 1\right) \notag \\
	&= \left(-1\right)y_{i}\left(x_{i} - 1\right)\left(-1\right)y_{j}\left(x_{j} - 1\right) \notag \\
	&= \left(-1\right)^{2}y_{i,j}\left(x_{i} - 1,x_{j} - 1\right).
\label{eq:twofactorminus}
\end{alignat}
%--------------------------------------------------
\section{Odd-Divisible Numbers: Different perspectives}
\label{s:odddivisiblenumbersdiffperspectives}
%--------------------------------------------------
Finally, we see the different possible perspectives for odd-divisible numbers.
\begin{alignat}{3}
	y_{i,j}\left(x_{i},x_{j}\right) &= 2\left(2x_{i}x_{j} + x_{i} + x_{j}\right) + 1 \notag \\
	&= 2\left(\left(2x_{i} + 1\right)x_{j} + x_{i}\right) + 1 \label{eq:oddnumbdiffpers1}\\
	\mathrm{respectively} \quad &= 2\left(\left(2x_{j} + 1\right)x_{i} + x_{j}\right) + 1
\label{eq:oddnumbdiffpers2}
\end{alignat}

We will use (\ref{eq:oddnumbdiffpers1}) respectively (\ref{eq:oddnumbdiffpers2}) in the next step, for the description of odd numbers which are not divisible by a particular other odd number.




















