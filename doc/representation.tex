\chapter{Representation of numbers}
\label{ch:representationofnumbers}
\minitoc
%-------------------------------------------------
I will introduce an easy represantion way for each times table of our given $x_{i,j}$, which we will finally use to find our primes.
%--------------------------------------------------
\section{What we already know}
\label{s:whatwealreadyknow}
%--------------------------------------------------
In section \ref{ch:odddivisiblenumbers}, I showed you a different possibility for describing odd-divisible numbers. We received the following
\begin{alignat}{3}
	x_{i,j} &:= 2x_{i}x_{j} + x_{i} + x_{j} \label{eq:xij_gen} \\
	&= \left(2x_{j} + 1\right)x_{i} + x_{j} \label{eq:xij_jconst} \\
	&= \left(2x_{i} + 1\right)x_{j} + x_{i} \label{eq:xij_iconst}	
\end{alignat}
with $x_{i}, x_{j} \in \mathbb{N}$.
%--------------------------------------------------
\section{Bit string representation}
\label{s:bitstringrepresantion}
%--------------------------------------------------
Let's take equation (\ref{eq:xij_iconst}). For example, its first numbers for $x_{1} = 1$ are
\begin{equation}
	x_{1,j} = 4, 7, 10, 13, 16, 19, 22, 25, 28, \dots
\label{eq:xi1_firstnumbers}\end{equation}

We can write this times table also as a bit string

\begin{table}[H]
\centering
\caption{Bit string representation for the case $x_{1} = 1$. In the second row we have given the belonging indices for each bit.}
\tiny
\begin{tabular}{cccccccccccccccccccccc||c}
				\cellcolor{yellow} 22 & 21 & 20 & \cellcolor{yellow} 19 & 18 & 17 & \cellcolor{yellow} 16 & 15 & 14 & \cellcolor{yellow} 13 & 12 & 11 & \cellcolor{yellow} 10 & 9 & 8 & \cellcolor{yellow} 7 & 6 & 5 & \cellcolor{yellow} 4 & 3 & 2 & 1 & \\
\hline				21 & 20 & 19 & 18 & 17 & 16 & 15 & 14 & 13 & 12 & 11 & 10 & 9  & 8 & 7 & 6 & 5 & 4 & 3 & 2 & 1 & 0 & Index \\
\hline\hline \rowcolor{green}	\cellcolor{red} 1 &  0 &  0 &  \cellcolor{red} 1 &  0 &  0 & \cellcolor{red} 1 &  0 &  0 & \cellcolor{red} 1 &  0 &  0 & \cellcolor{red} 1 & 0 & 0 & \cellcolor{red} 1 & 0 & 0 & \cellcolor{red} 1 & 0 & 0 & 0 & $x_{1,j}$
\end{tabular}
\label{tab:bitstringx1_1}
\end{table}

What you see, is a binary number with bit 1 (red) at all positions (yellow) for which we have a divisible number, else its bit is 0 (green). Like a binary number, this kind of representation starts at the right and goes to the left. Since we have an unending amount of natural numbers, this bit string is unending, too.

\vspace{0.3cm}
Later, we will see the reason for choosing 1 for divisible numbers instead of 0.

\vspace{0.3cm}
Its decimal notation for a fixed $x_{i}$ is given by
\begin{alignat}{3}
	x_{i=\mathrm{const.},j_{\left(10\right)}} := \sum_{x_{j} = 1} 2^{\left(2x_{i} + 1\right)x_{j} + x_{i} - 1} 
\label{eq:bitstringdec}\end{alignat} 












