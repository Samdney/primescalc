\chapter{The recursive calculation}
\label{ch:therecursivecalculation}
\minitoc
%--------------------------------------------------
In this section, now, we do the final recursive calculation. To understand the deep structure we will do this by discussing the first steps by manually calculation. We will see, it exists different ways how you can look at each situation/problem in each step, hence this will mainly a discussion of this different ways.
%--------------------------------------------------
%--------------------------------------------------
\section{Recursion step: $n^{\left(0\right)} = 0$}
\label{s:recursionstepn0_0}
%--------------------------------------------------
Let's start with our step zero. It's called zero, since we start with the simplest first possible prime generation. Since we know $3$ is the smallest odd number at all, we know $3$ has to be a prime. Hence we have our first prime.

\vspace{0.3cm}
\textbf{Results of step $n^{\left(0\right)} = 0$}:
\begin{itemize}
	\item Known prime $x_{i,j}$: $1$
	\item Known prime $y_{i,j}$: $3$
	\item Known prime number range for $x_{i,j}$: $[1,1]$
	\item Known prime number range for $y_{i,j}$: $[1,3]$
\end{itemize}

That's of course not much and trivial, but hey, it's a beginning!










%--------------------------------------------------
%\section{Recursion step: $n^{\left(0\right)} = 0$}
%\label{s:recursionstepn0_0}
%--------------------------------------------------
%We start our calculation with an easy consideration. We assume, we know $3=2\cdot 1 + 1$ and $5=2\cdot2 + 1$ are prime numbers. Hence, we know
%\begin{alignat}{3}
%	y_{1,j}^{\left(1\right)}\left(1,x_{j}^{\left(1\right)}\right) &= 2\left(3x_{j}^{\left(1\right)} + 1 - \mu\left(1\right)\right) + 1 \notag \\
%	\mu\left(1\right) &= 1, 2 \label{eq:eqsn0_0_1} \\
%	\mathrm{and} \quad y_{2,j}^{\left(2\right)}\left(2,x_{j}^{\left(2\right)}\right) &= 2\left(5x_{j}^{\left(2\right)} + 2 - \mu\left(2\right)\right) + 1 \notag \\
%	\mu\left(1\right) &= 1,2,3,4 \label{eq:eqsn0_0_2}
%\end{alignat}
%and want to calculate the intersection
%\begin{alignat}{3}
%	3x_{j}^{\left(1\right)} + 1 - \mu\left(1\right) = 5x_{j}^{\left(2\right)} + 2 - \mu\left(2\right).
%\label{eq:intn0_0}
%\end{alignat}

%Now, we can choose between different angle of views, how we solve this intersection and how we handle the whole situation for our next recursion step.

%\vspace{0.3cm}
%Version 1:

%The way to go ...
%\begin{enumerate}
%	\item We fix: The whole set of valid $\mu\left(1\right) - \mu\left(2\right)$ values is allowed.
%	\item We determine the solutions for $x_{j}^{\left(1\right)}$ respectively $x_{j}^{\left(2\right)}$ without consideration of any definition ranges for $x_{j}$'s and the final range of interest.
%	\item ...
%	\item ...
%\label{en:n0_0_v1way}
%\end{enumerate}

%The calculation ...
%\begin{enumerate}
%	\item Trivial: $\mu\left(1\right) - \mu\left(2\right) = \{1,2\} - \{1,2,3,4\} = \{-3,\dots,1\}$
%	\item 
%		\begin{alignat}{3}
%			3x_{j}^{\left(1\right)} - 5x_{j}^{\left(2\right)} &= 1 + \mu\left(1\right) - \mu\left(2\right) \notag \\
%			x_{j}^{\left(1\right)} - x_{j}^{\left(2\right)} &= \frac{1}{3}\left(2x_{j}^{\left(2\right)} + 1 + \mu\left(1\right) - \mu\left(2\right)\right) \label{eq:intn0_0_v1calc0}
%		\end{alignat}
	
%		The right side of (\ref{eq:intn0_0_v1calc0}) has to be an integer. We can directly see one valid soulution. Be
%		\begin{equation}
%			x_{j}^{\left(2\right)} = 1 + \mu\left(1\right) - \mu\left(2\right)
%		\label{eq:intn0_0_v1calc1}
%		\end{equation}
		
%		which we put into (\ref{eq:intn0_0_v1calc0}) and receive
%		\begin{equation}
%			 x_{j}^{\left(1\right)} = 2\left(1 + \mu\left(1\right) - \mu\left(2\right)\right).
%		\label{eq:intn0_0_v1calc2}
%		\end{equation}

%		Finally, we have all possible solutions for
%		\begin{alignat}{3}
%			x_{j}^{\left(1\right)} &= 5z^{\left(1,2\right)} + 2\left(1 + \mu\left(1\right) - \mu\left(2\right)\right) \label{eq:int0_0_v1calcsol1}\\
%			x_{j}^{\left(2\right)} &= 3z^{\left(1,2\right)} + 1\left(1 + \mu\left(1\right) - \mu\left(2\right)\right), \label{eq:int0_0_v1calcsol2}
%		\end{alignat}
%		$z^{\left(1,2\right)} \in \mathbb{Z}$.

%	\item ...
%	\item ...
%\label{en:n0_0_v1calc}
%\end{enumerate}
