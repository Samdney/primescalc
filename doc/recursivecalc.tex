\chapter{The recursive calculation}
\label{ch:therecursivecalculation}
\minitoc
%--------------------------------------------------
In this section, now, we do the final recursive calculation. To understand the deep structure we will do this by discussing the first steps by manually calculation. We will see, it exists different ways how you can look at each situation/problem in each step, hence this will mainly a discussion of this different ways.
%--------------------------------------------------
%--------------------------------------------------
\section{Recursion step: $n^{\left(0\right)} = 0$}
\label{s:recursionstepn0_0}
%--------------------------------------------------
%--------------------------------------------------
\subsection{Calculation}
\label{ss:calcualtionstepn0_0}
%--------------------------------------------------
Let's start with our step zero. It's called zero, since we start with the simplest first possible prime generation. Since we know $3$ is the smallest odd number at all, we know $3$ has to be a prime. Hence we have our first prime.
%--------------------------------------------------
\subsection{Results}
\label{ss:resultsstepn0_0}
%--------------------------------------------------
\begin{itemize}
	\item Known prime $x_{i,j}$: $1$
	\item Known prime $y_{i,j}$: $3$
	\item Known prime number range for $x_{i,j}$: $[1,1]$
	\item Known prime number range for $y_{i,j}$: $[1,3]$
\end{itemize}

That's of course not much and trivial, but hey, it's a beginning!
%--------------------------------------------------
\section{Recursion step: $n^{\left(0\right)} = 1$}
\label{s:recursionstepn0_1}
%--------------------------------------------------
%--------------------------------------------------
\subsection{Calculation}
\label{ss:calcualtionstepn0_1}
%--------------------------------------------------
Now, we can start with our first step. From $n^{\left(0\right)} = 0$, we know $3(1)$.

\vspace{0.3cm}
\textbf{Notation Note:} From now on I will write $y\left(x\right)$ for shortness. For example, $3(1)$ means $3 = 2\cdot 1 + 1$. If I only write $1$, I mean $x = 1$.

\vspace{0.3cm}
So, let's look again at the values for $x_{1,j}$ from 
\begin{equation}
	y_{1,j}\left(1,x_{j}\right) = 2\left(3x_{j} + 1\right) + 1, \quad x_{1,j} = 3x_{j} + 1.
\label{eq:stepn0_1_valuesfordiv3}
\end{equation}

\begin{table}[H]
\centering
\caption{The first values for $x_{1,j}$ are marked in bold. Italic values are within the not describable range.}
\begin{tabular}{c|ccccccccccccccccccc}
	$x_{j}$ & -- & -- & -- & 1 & -- & -- & 2 & --  & -- & 3 & -- & -- & 4 & -- & -- & 5 & -- & -- & 6 \\
\hline	$x_{1,j}$ & \textit{1} & 2 & 3 & \textbf{4} & 5 & 6 & \textbf{7} & 8 & 9 & \textbf{10} & 11 & 12 & \textbf{13} & 14 & 15 & \textbf{16} & 17 & 18 & \textbf{19}
\end{tabular}
\label{tab:stepn0_1_valuesfordiv3}
\end{table}

What we can see from our table \ref{tab:stepn0_1_valuesfordiv3} are the next two sure primes, the numbers $5(2)$ and $7(3)$. All larger numbers could theoretically still have a divider, hence we only know this two additional numbers surely, at the moment.

\vspace{0.3cm}
We can describe this two numbers with (\ref{eq:repoddnotdivnumb}) and (\ref{eq:defmu}) by

\begin{equation}
	x\left(1,1\right) = 3\cdot 1 + 1 - \mu\left(1\right),
\label{eq:stepn0_1_eqfordiv3res}
\end{equation}
with
\begin{equation}
	\mu\left(1\right) = 1,2.
\label{eq:stepn0_1_eqfordiv3res}
\end{equation}
and a maximum value for $x_{j} = 1$.
%--------------------------------------------------
\subsection{Results}
\label{ss:resultsstepn0_1}
%--------------------------------------------------
\textbf{Only from step $n^{\left(0\right)} = 1$}:
\begin{itemize}
	\item Known prime $x_{i,j}$: $2,3$
	\item Known primes $y_{i,j}$: $5,7$
	\item Known prime number range for $x_{i,j}$: $[2,4]$
	\item Known prime number range for $y_{i,j}$: $[5,9]$
\end{itemize}

\textbf{From all steps until now}:
\begin{itemize}
	\item Known prime $x_{i,j}$: $1,2,3$
	\item Known primes $y_{i,j}$: $3,5,7$
	\item Known prime number range for $x_{i,j}$: $[1,4]$
	\item Known prime number range for $y_{i,j}$: $[1,9]$
\end{itemize}

%--------------------------------------------------
\section{Recursion step: $n^{\left(0\right)} = 2$}
\label{s:recursionstepn0_2}
%--------------------------------------------------
%--------------------------------------------------
\subsection{Calculation}
\label{ss:calcualtionstepn0_2}
%--------------------------------------------------
Now, from our steps above we have 
\begin{alignat}{3}
	x\left(1,x_{j}^{\left(1\right)}\right) &= 3\cdot x_{j}^{\left(1\right)} + 1 - \Delta x_{i}^{\left(a,b\right)}\mu\left(1\right) \label{eq:stepn0_2_eq3} \\
	x\left(2,x_{j}^{\left(2\right)}\right) &= 5\cdot x_{j}^{\left(2\right)} + 2 - \Delta x_{i}^{\left(a,b\right)}\mu\left(2\right) \label{eq:stepn0_2_eq5} \\
	x\left(3,x_{j}^{\left(3\right)}\right) &= 7\cdot x_{j}^{\left(3\right)} + 3 - \Delta x_{i}^{\left(a,b\right)}\mu\left(3\right), \label{eq:stepn0_2_eq7}
\end{alignat} 

with $\mu\left(1\right) = 1,2$, $\mu\left(2\right) = 1,2,3,4$, $\mu\left(3\right) = 1,2,3,4,5,6$ and $a,b \in \{1,2,3\}$, $a\neq b$. Let's have a look at the values (including $\mu = 0$).

\begin{table}[H]
\centering
\caption{The first values for $x_{i,j}$ are marked in bold. Italic values are within the not describable range.}
\begin{tabular}{c|ccccccccccccccccccc}
\hline	$x_{1,j}$ & \textit{1} & 2 & 3 & \textbf{4} & 5 & 6 & \textbf{7} & 8 & 9 & \textbf{10} & 11 & 12 & \textbf{13} & 14 & 15 & \textbf{16} & 17 & 18 & \textbf{19} \\
	$x_{2,j}$ & \textit{1} & \textit{2} & 3 & 4 & 5 & 6 & \textbf{7} & 8 & 9 & 10 & 11 & \textbf{12} & 13 & 14 & 15 & 16 & \textbf{17} & 18 & 19 \\
	$x_{3,j}$ & \textit{1} & \textit{2} & \textit{3} & 4 & 5 & 6 & 7 & 8 & 9 & \textbf{10} & 11 & 12 & 13 & 14 & 15 & 16 & \textbf{17} & 18 & 19
	%$x_{5,j}$ & \textit{1} & \textit{2} & \textit{3} & \textit{4} & \textit{5} & 6 & 7 & 8 & 9 & 10 & 11 & 12 & 13 & 14 & 15 & \textbf{16} & 17 & 18 & 19
\end{tabular}
\label{tab:stepn0_2_values357}
\end{table}

Now, at this point, we have different possible ways, how we can calculate more primes from this. We will discuss this different ways step by step.\\
At first, we make an intersection between (\ref{eq:stepn0_2_eq3}) and (\ref{eq:stepn0_2_eq5}), $\Delta x_{i}^{\left(1,2\right)} = 1$,
\begin{alignat}{3}
	0 &= 3\cdot x_{j}^{\left(1\right)} - 5\cdot x_{j}^{\left(2\right)} + 1 - 2 - \mu\left(1\right)  + \mu\left(2\right) \notag \\
	  &= 3\cdot x_{j}^{\left(1\right)} - 5\cdot x_{j}^{\left(2\right)} - 1 - \mu\left(1\right)  + \mu\left(2\right). \label{eq:step0_2_int35}
\end{alignat}

From the main section prior to this, we know our solutions with (\ref{eq:solxj1delta1_all}) and (\ref{eq:solxj2delta1_all}).
\begin{alignat}{3}
	x_{j}^{\left(1\right)} &= 5z^{\left(1,2\right)} + \left(1 + \mu\left(1\right) - \mu\left(2\right)\right)2 \label{eq:step0_2_sol35_1} \\
	x_{j}^{\left(2\right)} &= 3z^{\left(1,2\right)} + \left(1 + \mu\left(1\right) - \mu\left(2\right)\right)1 \label{eq:step0_2_sol35_2},
\end{alignat}
$z^{\left(1,2\right)} \in \mathbb{Z}$.\\

Now, let's have a look at the intersection between (\ref{eq:stepn0_2_eq3}) and (\ref{eq:stepn0_2_eq7}), $\Delta x_{i}^{\left(1,3\right)} = 2$,
\begin{alignat}{3}
	0 &= 3\cdot x_{j}^{\left(1\right)} - 7\cdot x_{j}^{\left(2\right)} + 1 - 3 - 2\mu\left(1\right)  + 2\mu\left(3\right) \notag \\
	  &= 3\cdot x_{j}^{\left(1\right)} - 7\cdot x_{j}^{\left(2\right)} - 2 - 2\mu\left(1\right)  + 2\mu\left(3\right). \label{eq:step0_2_int37}
\end{alignat}

Our solutions with (\ref{eq:solxj1deltagen_all}) and (\ref{eq:solxj2deltagen_all}) are
\begin{alignat}{3}
	x_{j}^{\left(1\right)} &= 7z^{\left(1,2\right)} + \left(1 + \mu\left(1\right) - \mu\left(3\right)\right)3 \label{eq:step0_2_sol35_1} \\
	x_{j}^{\left(2\right)} &= 3z^{\left(1,2\right)} + \left(1 + \mu\left(1\right) - \mu\left(3\right)\right)1 \label{eq:step0_2_sol35_2},
\end{alignat}
$z^{\left(1,2\right)} \in \mathbb{Z}$.\\




% ======
%If we look at the values of all three equations, we see that our maximum valid range is now $[4,10]$ given by (\ref{eq:stepn0_2_eq7}) with $x_{j}^{\left(3\right)} = 1$.\\
% ======











