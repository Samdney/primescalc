\chapter{The recursive calculation}
\label{ch:therecursivecalculation}
\minitoc
%--------------------------------------------------
In this section, we do the final recursive calculation. To understand the deep structure we will do this by discussing the first steps by manually calculation. We will see, it exists different ways how you can look at each situation/problem in each step, hence this will mainly a discussion of this different ways.
%--------------------------------------------------
%--------------------------------------------------
\section{Recursion step: $n^{\left(0\right)} = 0$}
\label{s:recursionstepn0_0}
%--------------------------------------------------
%--------------------------------------------------
\subsection{Calculation}
\label{ss:calcualtionstepn0_0}
%--------------------------------------------------
Let's start with our step zero. It's called zero, since we start with the simplest first possible prime generation. Since we know $3$ is the smallest odd number at all, we know $3$ has to be a prime. Hence we have our first prime.
%--------------------------------------------------
\subsection{Results}
\label{ss:resultsstepn0_0}
%--------------------------------------------------
\begin{itemize}
	\item Known prime $x_{i,j}$: $1$
	\item Known prime $y_{i,j}$: $3$
	\item Known prime number range for $x_{i,j}$: $[1,1]$
	\item Known prime number range for $y_{i,j}$: $[1,3]$
\end{itemize}

That's of course not much and trivial, but hey, it's a beginning!
%--------------------------------------------------
\section{Recursion step: $n^{\left(0\right)} = 1$}
\label{s:recursionstepn0_1}
%--------------------------------------------------
%--------------------------------------------------
\subsection{Calculation}
\label{ss:calcualtionstepn0_1}
%--------------------------------------------------
Now, we can start with our first step. From $n^{\left(0\right)} = 0$, we know $3(1)$.

\vspace{0.3cm}
\textbf{Notation Note:} From now on I will write $y\left(x\right)$ for shortness. For example, $3(1)$ means $3 = 2\cdot 1 + 1$. If I only write $1$, I mean $x = 1$.

\vspace{0.3cm}
So, let's look again at the values for $x_{1,j}$ from 
\begin{equation}
	y_{1,j}\left(1,x_{j}\right) = 2\left(3x_{j} + 1\right) + 1, \quad x_{1,j} = 3x_{j} + 1.
\label{eq:stepn0_1_valuesfordiv3}
\end{equation}

\begin{table}[H]
\centering
\caption{The first values for $x_{1,j}$ are marked in bold. Italic values are within the not describable range.}
\begin{tabular}{c|ccccccccccccccccccc}
	$x_{j}$ & -- & -- & -- & 1 & -- & -- & 2 & --  & -- & 3 & -- & -- & 4 & -- & -- & 5 & -- & -- & 6 \\
\hline	$x_{1,j}$ & \textit{1} & 2 & 3 & \textbf{4} & 5 & 6 & \textbf{7} & 8 & 9 & \textbf{10} & 11 & 12 & \textbf{13} & 14 & 15 & \textbf{16} & 17 & 18 & \textbf{19}
\end{tabular}
\label{tab:stepn0_1_valuesfordiv3}
\end{table}

What we can see from our table \ref{tab:stepn0_1_valuesfordiv3} are the next two sure primes, the numbers $5(2)$ and $7(3)$. All larger numbers could theoretically still have a divider, hence we only know this two additional numbers surely, at the moment.

\vspace{0.3cm}
We can describe this two numbers with (\ref{eq:repoddnotdivnumb}) and (\ref{eq:defmu}) by

\begin{equation}
	x_{1,1}\left(1,1\right) = 3\cdot 1 + 1 - \mu\left(1\right),
\label{eq:stepn0_1_eqfordiv3res}
\end{equation}
with
\begin{equation}
	\mu\left(1\right) = 1,2.
\label{eq:stepn0_1_eqfordiv3res}
\end{equation}
and a maximum value for $x_{j} = 1$.
%--------------------------------------------------
\subsection{Results}
\label{ss:resultsstepn0_1}
%--------------------------------------------------
\textbf{Only from step $n^{\left(0\right)} = 1$}:
\begin{itemize}
	\item Known prime $x_{i,j}$: $2,3$
	\item Known primes $y_{i,j}$: $5,7$
	\item Known prime number range for $x_{i,j}$: $[2,4]$
	\item Known prime number range for $y_{i,j}$: $[5,9]$
	\item Describing prime equation: $x_{1,1}\left(1,1\right) = 3\cdot 1 + 1 - \mu\left(1\right)$, $\mu\left(1\right) = 1,2$
\end{itemize}

\textbf{From all steps until now}:
\begin{itemize}
	\item Known prime $x_{i,j}$: $1,2,3$
	\item Known primes $y_{i,j}$: $3,5,7$
	\item Known prime number range for $x_{i,j}$: $[1,4]$
	\item Known prime number range for $y_{i,j}$: $[1,9]$
	\item Describing prime equation for $x_{i,j} \in [2,4]$: $x_{1,1}\left(1,1\right) = 3\cdot 1 + 1 - \mu\left(1\right)$, $\mu\left(1\right) = 1,2$
\end{itemize}

Our first prime equation for $x_{i,j} \in [2,4]$:
\begin{alignat}{3}
	x_{1,1}\left(1,1,\mu\left(1\right)\right) = \left(2 \cdot 1 + 1\right)1 + 1 - \mu\left(1\right) \label{eq:resultstepn0_1}
\end{alignat}
%--------------------------------------------------
\section{Recursion step: $n^{\left(0\right)} = 2$}
\label{s:recursionstepn0_2}
%--------------------------------------------------
%--------------------------------------------------
\subsection{Calculation}
\label{ss:calcualtionstepn0_2}
%--------------------------------------------------
Now, from our steps above we have (\ref{eq:resultstepn0_1}) and hence 
\begin{alignat}{3}
	x_{1,j}\left(1,x_{j}^{\left(1\right)},\mu\left(1\right)\right) =& \ 3x_{j}^{\left(1\right)} + 1 - \Delta x_{i}^{\left(a,b\right)}\mu\left(1\right) \notag \\ 
	=& \ \left(2 \cdot 1 + 1\right)x_{j}^{\left(1\right)} + 1 - \Delta x_{i}^{\left(a,b\right)}\mu\left(1\right) \label{eq:stepn0_2_eq3} \\
	x_{2,j}\left(2,x_{j}^{\left(2\right)},\mu\left(2\right)\right) =& \ 5x_{j}^{\left(2\right)} + 2 - \Delta x_{i}^{\left(a,b\right)}\mu\left(2\right) \notag \\
	=& \ \left(2 \cdot 2 + 1\right)x_{j}^{\left(2\right)} + 2 - \Delta x_{i}^{\left(a,b\right)}\mu\left(2\right) \notag \\
	=& \ \left(2 \cdot \left(\left(2 \cdot 1 + 1\right)1 + 1 - 2\right) + 1\right)x_{j}^{\left(2\right)} \notag \\
	&+ \left(\left(2 \cdot 1 + 1\right)1 + 1 - 2\right) - \Delta x_{i}^{\left(a,b\right)}\mu\left(2\right) \label{eq:stepn0_2_eq5} \\
	x_{3,j}\left(3,x_{j}^{\left(3\right)},\mu\left(3\right)\right) =& \ 7x_{j}^{\left(3\right)} + 3 - \Delta x_{i}^{\left(a,b\right)}\mu\left(3\right) \notag \\
	=& \ \left(2 \cdot 3 + 1\right)x_{j}^{\left(3\right)} + 3 - \Delta x_{i}^{\left(a,b\right)}\mu\left(3\right) \notag \\
	=& \ \left(2 \cdot \left(\left(2 \cdot 1 + 1\right)1 + 1 - 1\right) + 1\right)x_{j}^{\left(3\right)} \notag \\
	&+ \left(\left(2 \cdot 1 + 1\right)1 + 1 - 1\right) - \Delta x_{i}^{\left(a,b\right)}\mu\left(3\right), \label{eq:stepn0_2_eq7}
\end{alignat} 

with $\mu\left(1\right) = 1,2$, $\mu\left(2\right) = 1,2,3,4$, $\mu\left(3\right) = 1,2,3,4,5,6$ and $a,b \in \{1,2,3\}$, $a\neq b$. Let's have a look at the values (including $\mu = 0$).

\begin{table}[H]
\centering
\caption{The first values for $x_{i,j}$ are marked in bold. Italic values are within the not describable range.}
\begin{tabular}{c|ccccccccccccccccccc}
\hline	$x_{1,j}$ & \textit{1} & 2 & 3 & \textbf{4} & 5 & 6 & \textbf{7} & 8 & 9 & \textbf{10} & 11 & 12 & \textbf{13} & 14 & 15 & \textbf{16} & 17 & 18 & \textbf{19} \\
	$x_{2,j}$ & \textit{1} & \textit{2} & 3 & 4 & 5 & 6 & \textbf{7} & 8 & 9 & 10 & 11 & \textbf{12} & 13 & 14 & 15 & 16 & \textbf{17} & 18 & 19 \\
	$x_{3,j}$ & \textit{1} & \textit{2} & \textit{3} & 4 & 5 & 6 & 7 & 8 & 9 & \textbf{10} & 11 & 12 & 13 & 14 & 15 & 16 & \textbf{17} & 18 & 19
	%$x_{5,j}$ & \textit{1} & \textit{2} & \textit{3} & \textit{4} & \textit{5} & 6 & 7 & 8 & 9 & 10 & 11 & 12 & 13 & 14 & 15 & \textbf{16} & 17 & 18 & 19
\end{tabular}
\label{tab:stepn0_2_values357}
\end{table}

Now, at this point, we have different possible ways, how we can calculate more primes from this. We will discuss this different ways step by step.\\
At first, we make an intersection between (\ref{eq:stepn0_2_eq3}) and (\ref{eq:stepn0_2_eq5}), $\Delta x_{i}^{\left(1,2\right)} = 1$,
\begin{alignat}{3}
	0 &= 3x_{j}^{\left(1\right)} - 5x_{j}^{\left(2\right)} + 1 - 2 - \mu\left(1\right)  + \mu\left(2\right) \notag \\
	  &= 3x_{j}^{\left(1\right)} - 5x_{j}^{\left(2\right)} - 1 - \mu\left(1\right)  + \mu\left(2\right). \label{eq:step0_2_int35}
\end{alignat}

From the main section prior to this, we know our solutions with (\ref{eq:solxj1delta1_all}) and (\ref{eq:solxj2delta1_all}).
\begin{alignat}{3}
	x_{j}^{\left(1\right)} &= 5z^{\left(1,2\right)} + \left(1 + \mu\left(1\right) - \mu\left(2\right)\right)2 \label{eq:step0_2_sol35_1} \\
	x_{j}^{\left(2\right)} &= 3z^{\left(1,2\right)} + \left(1 + \mu\left(1\right) - \mu\left(2\right)\right)1 \label{eq:step0_2_sol35_2},
\end{alignat}
$z^{\left(1,2\right)} \in \mathbb{Z}$ and finally
\begin{alignat}{3}
	x_{1,j}^{\left(1,2\right)} &= 3 \cdot 5 z^{\left(1,2\right)} + 3\left(1 + \mu\left(1\right) - \mu\left(2\right)\right)2 + 1 - \mu\left(1\right) \label{eq:step0_2_sol35_x1j_12_l1} \\
	&= 3 \cdot 5 z^{\left(1,2\right)} + 7 + 5\mu\left(1\right) - 6\mu\left(2\right) \label{eq:step0_2_sol35_x1j_12_l2} \\
	x_{2,j}^{\left(1,2\right)} &= 5 \cdot 3 z^{\left(1,2\right)} + 5\left(1 + \mu\left(1\right) - \mu\left(2\right)\right)1 + 2 - \mu\left(2\right) \label{eq:step0_2_sol35_x2j_12_l1} \\
	&= 3 \cdot 5 z^{\left(1,2\right)} + 7 + 5\mu\left(1\right) - 6\mu\left(2\right). \notag %\label{eq:step0_2_sol35_x2j_12_l2}
\end{alignat}

Ok, we have our equation which generates all solutions for our intersection. But since we only want to generate primes with this equation, we have to restrict the valid range. So, what is the maximum valid range for $x_{i,j}$ after the intersection above?

\begin{enumerate}
	\item \textit{Lower bound}: This bound is easy to see. We know for each original equation of the form $x_{i,j}\left(x_{i},x_{j},\mu\left(x_{i}\right)\right) = \left(2x_{i} + 1\right)x_{j} + x_{i} - \Delta x_{i}^{\left(a,b\right)}\mu\left(x_{i}\right)$ that its lower bound is given by 
	
	\begin{alignat}{3}
	x_{i,j}\left(x_{i},0,0\right) &= \left(2x_{i} + 1\right) \cdot 0 + x_{i} - 0  + 1 \label{eq:lowerbound1} \\
	&= x_{i} + 1 \label{eq:lowerbound2} \\
	\Leftrightarrow \quad x_{i,j}\left(x_{i},1,2x_{i}\right) &= \left(2x_{i} + 1\right) \cdot 1 + x_{i} - 2x_{i}. \label{eq:lowerbound3}
	\end{alignat}
	
	If follows, for the intersection of $n$ equations for the lower bound
	\begin{alignat}{3}
		x_{i,j}^{\left(1 \dots n\right), \ \mathrm{lower \ bound}}\left(x_{i}^{\mathrm{max}},1,2x_{i}^{\mathrm{max}}\right) &= \left(2x_{i}^{\mathrm{max}} + 1\right) \cdot 1 + x_{i}^{\mathrm{max}} - 2x_{i}^{\mathrm{max}} \notag \\
		&= x_{i}^{\mathrm{max}} + 1, \label{eq:lowerbound_neqs}
	\end{alignat}
	with $x_{i}^{\mathrm{max}} := \mathrm{max}\{ x_{i}^{\left(1\right)}, x_{i}^{\left(2\right)}, \dots, x_{i}^{\left(n-1\right)}, x_{i}^{\left(n\right)} \}$.

	\item \textit{Upper bound}: We have again the intersection of $n$ equations. As first assumption we can set
	
	\begin{alignat}{3}
		x_{i,j}\left(x_{i}^{\mathrm{max}} + 1,1,0\right) &= \left(2\left(x_{i}^{\mathrm{max}} + 1\right) + 1\right) \cdot 1 + \left(x_{i}^{\mathrm{max}} + 1\right), \label{eq:firstassumption}
	\end{alignat}
	with $x_{i}^{\mathrm{max}} := \mathrm{max}\{ x_{i}^{\left(1\right)}, x_{i}^{\left(2\right)}, \dots, x_{i}^{\left(n-1\right)}, x_{i}^{\left(n\right)} \}$. But if we look closer on our multiplication tables, we see,
	\begin{alignat}{3}
	x_{i,j}\left(x_{i}^{\mathrm{max}} + 1,1,0\right) &= \left(2\left(x_{i}^{\mathrm{max}} + 1\right) + 1\right) \cdot 1 + \left(x_{i}^{\mathrm{max}} + 1\right) \notag \\
	&= \left(2 \cdot 1 + 1\right)\left(x_{i}^{\mathrm{max}} + 1\right) + 1, \label{eq:firstvalue}
	\end{alignat}
	that this values are also always an element of the multiplication table of three. Since we always use the three multiplication table equation for intersection calculation, we have this first values of every other multipliation table already given. This means, we can look for an higher upper bound. \\

	Let's go to our next assumption and take the second value
	\begin{alignat}{3}
		x_{i,j}\left(x_{i}^{\mathrm{max}} + 1,2,0\right) &= \left(2\left(x_{i}^{\mathrm{max}} + 1\right) + 1\right) \cdot 2 + \left(x_{i}^{\mathrm{max}} + 1\right) \notag \\
	&= \left(2 \cdot 2 + 1\right)\left(x_{i}^{\mathrm{max}} + 1\right) + 2. \label{eq:secondassumption}
	\end{alignat}

	Assume we already know all primes in the range $x_{i} \in [1,x_{i}^{\mathrm{s}}]$ after the s'th, $s \in \mathbb{N}_{0}$, recursion step, with $n$ equations. From this assumptions we can follow our valid maximum upper bound for the generation of the next primes
	\begin{alignat}{3}
		x_{i,j}^{\left(1 \dots n\right), \ \mathrm{upper \ bound}}\left(x_{i}^{\mathrm{max}}+1,x_{i}^{\mathrm{max}} + 1,0\right) =& \ \left(2\left(x_{i}^{\mathrm{max}}+1\right) + 1\right)\left(x_{i}^{\mathrm{max}}+1\right) \notag \\
	&+ \left(x_{i}^{\mathrm{max}}+1\right) - 0 -1 \notag \\
	=& \ 2\left(x_{i}^{\mathrm{max}}\right)^{2} + 6x_{i}^{\mathrm{max}} + 3, \label{eq:upperbound}
	\end{alignat}
	with $x_{i}^{\mathrm{max}} = x_{i}^{\mathrm{s}}$. Finally, we determine the value of $x_{j}$ for an arbitrary multiplication table $x_{i}$ for our upper bound.
	\begin{alignat}{3}
		\left(2x_{i} + 1\right)x_{j} + x_{i} &= 2\left(x_{i}^{\mathrm{max}}\right)^{2} + 6x_{i}^{\mathrm{max}} + 3 \notag \\
		\left(2x_{i} + 1\right)x_{j} &= 2\left(x_{i}^{\mathrm{max}}\right)^{2} + 6x_{i}^{\mathrm{max}} + 3 - x_{i} \notag \\
		x_{j}^{\mathrm{upper \ bound}} &:= \left\lfloor \frac{1}{2x_{i} + 1}\left(2\left(x_{i}^{\mathrm{max}}\right)^{2} + 6x_{i}^{\mathrm{max}} + 3 - x_{i}\right) \right\rfloor \label{eq:xi_upperbound_2xi+1table}
	\end{alignat}	

\label{en:intbounds35_dis}
\end{enumerate}

Since we know our maximum valid range for our first intersection, we can think how we can receive the belonging valid $z$ values as a function of all $\mu$'s, now.\\

\textbf{Notation:} Given by (\ref{eq:upperbound}), we see that our true maximum upper bound $x_{i,j}^{\mathrm{upper \ bound}}$ in the recursion step $n^{\left(0\right)} = 1$ is already $x_{1,j}^{\left(1\right), \ \mathrm{upper \ bound}} = 2 \cdot 1^{2} + 6 \cdot 1 + 3 = 11$ and not $4$, what we claimed before.\\

What we will need additionally, is the connection between the $z$-values for the equations of the same multiplication table $x_{i}$, but with different $\Delta x_{i}^{\left(a,b\right)}$-values.
\begin{alignat}{3}
	x_{j} &= \left(2x_{i} + 1\right)z + x_{i} + \Delta x_{i}^{\left(a,b\right)}\mu\left(x_{i}\right) \notag \\
	x_{j} &= \left(2x_{i} + 1\right)z^{\prime} + x_{i} + \left(\Delta x_{i}^{\left(a,b\right)}\right)^{\prime}\mu\left(x_{i}\right) \notag \\ 
	\Rightarrow \quad 0 &= \left(2x_{i} + 1\right)\left(z - z^{\prime}\right) + \Delta x_{i}^{\left(a,b\right)}\mu\left(x_{i}\right) - \left(\Delta x_{i}^{\left(a,b\right)}\right)^{\prime}\mu\prime\left(x_{i}\right) \label{eq:differentDeltaxforsamexi}
	%\Leftrightarrow \quad z - z^{\prime} &= - \frac{\Delta x_{i}^{\left(a,b\right)}\mu\left(x_{i}\right) - \left(\Delta x_{i}^{\left(a,b\right)}\right)^{\prime}\mu\prime\left(x_{i}\right)}{2x_{i} + 1}\label{eq:differentDeltaxforsamexi}
\end{alignat}






%Let's go back to our original problem and the solutions (\ref{eq:step0_2_sol35_1}) and (\ref{eq:step0_2_sol35_2}) and the final $x_{i,j}^{\left(1,2\right)}$ equation (\ref{eq:step0_2_sol35_x1j_12_l1}) - (\ref{eq:step0_2_sol35_x2j_12_l1}).\\

%On the one hand, from (\ref{eq:lowerbound_neqs}) and (\ref{eq:upperbound}) respectively (\ref{eq:xi_upperbound_2xi+1table}) we know our valid parameter ranges. On the other hand, if we look at our solution equations for the intersection, we see that the valid values for $z^{\left(1,2\right)}$ are difficult to determine. So, this is the right point to introduce our shifting modification from the prior section. We have given
%\begin{alignat}{3}
%	x_{1,j}^{\left(1\right)}\left(1, x_{j}^{\left(1\right)}\right) &= 3\left(x_{j}^{\left(1\right)} + s_{j}^{\left(1\right)}\right) + 1 - \mu\left(1\right) \label{eq:eq1_shifted_35_3} \\
%	x_{2,j}^{\left(2\right)}\left(2, x_{j}^{\left(2\right)}\right) &= 5\left(x_{j}^{\left(2\right)} + s_{j}^{\left(2\right)}\right) + 2 - \mu\left(2\right) \label{eq:eq2_shifted_35_5}
%\end{alignat}
%and
%\begin{alignat}{3}
%	x_{j,s}^{\left(1\right)} &:= x_{j}^{\left(1\right)} + s_{j}^{\left(1\right)} &= 5z^{\left(1,2\right)} + \left(1 + \mu\left(1\right) - \mu\left(2\right)\right)2 \label{eq:step1_sol_xj1_shifted_35} \\
%	x_{j,s}^{\left(2\right)} &:= x_{j}^{\left(2\right)} + s_{j}^{\left(2\right)} &= 3z^{\left(1,2\right)} + \left(1 + \mu\left(1\right) - \mu\left(2\right)\right)1. \label{eq:step1_sol_xj2_shifted_25}
%\end{alignat}

%We start with the easier one of the last two equations, equation (\ref{eq:step1_sol_xj2_shifted_25}). Here, we have the term $\mu\left(1\right) - \mu\left(2\right)$, which makes it difficult to find our valid range of $z^{\left(1,2\right)}$ for given $x_{j}$. Hence, wie define an appropriate value for $s_{j}^{\left(2\right)}$, which can help us to solve this problem. We define
%\begin{alignat}{3}
%	s_{j}^{\left(2\right)} &:= -\mu\left(2\right).\label{eq:def_sj2_mu2}
%\end{alignat}
%With this follows
%\begin{alignat}{3}
%	x_{j}^{\left(2\right)} &= 3z^{\left(1,2\right)} + 1 + \mu\left(1\right) \label{eq:def_sj2_mu2_sol} \\
%	&= 3\left(z^{\left(1,2\right)} + 1\right) + 1 - \mu\left(1\right). \label{eq:def_sj2_mu2_sol_shiftedZmod}
%\end{alignat}
%Equation (\ref{eq:def_sj2_mu2_sol}) is very familiar for us, apart from the sign in the last term. In (\ref{eq:def_sj2_mu2_sol_shiftedZmod}), we brought to our well known shape from our initial definition. Since $z^{\left(1,2\right)}$ is definied on $\mathbb{Z}$ this isn't a problem. But have attention, that through this shape switching the meaning of the relationship to the multiplication table values, the "impact direction of $\mu$", changed.
















% ======
%Let's look at (\ref{eq:step0_2_sol35_2}) again
%\[ x_{j}^{\left(2\right)} = 3z^{\left(1,2\right)} + \left(1 + \mu\left(1\right) - \mu\left(2\right)\right)1. \]
% ======

% ======
%From this follows for our current example with $x_{i}^{\mathrm{max}} := 2$,
%\begin{alignat}{3}
%	x_{j}^{\left(1\right), \mathrm{upper \ bound}} &= \left\lfloor \frac{2}{3}\left(2^{2} + 1\right) \right\rfloor + 2 \cdot 2 \notag \\
%	&= \left\lfloor \frac{10}{3} \right\rfloor + 4 \notag \\
%	&= 7. \label{eq:upperbound35}
%\end{alignat}
% ======

% ======
%Let's have a look at the intersection between (\ref{eq:stepn0_2_eq3}) and (\ref{eq:stepn0_2_eq7}), $\Delta x_{i}^{\left(1,3\right)} = 2$,
%\begin{alignat}{3}
%	0 &= 3x_{j}^{\left(1\right)} - 7x_{j}^{\left(2\right)} + 1 - 3 - 2\mu\left(1\right)  + 2\mu\left(3\right) \notag \\
%	  &= 3x_{j}^{\left(1\right)} - 7x_{j}^{\left(2\right)} - 2 - 2\mu\left(1\right)  + 2\mu\left(3\right). \label{eq:step0_2_int37}
%\end{alignat}

%Our solutions with (\ref{eq:solxj1deltagen_all}) and (\ref{eq:solxj2deltagen_all}) are
%\begin{alignat}{3}
%	x_{j}^{\left(1\right)} &= 7z^{\left(1,2\right)} + \left(1 + \mu\left(1\right) - \mu\left(3\right)\right)3 \label{eq:step0_2_sol35_1} \\
%	x_{j}^{\left(2\right)} &= 3z^{\left(1,2\right)} + \left(1 + \mu\left(1\right) - \mu\left(3\right)\right)1 \label{eq:step0_2_sol35_2},
%\end{alignat}
%$z^{\left(1,2\right)} \in \mathbb{Z}$.\\
% ======

% ======
%If we look at the values of all three equations, we see that our maximum valid range is now $[4,10]$ given by (\ref{eq:stepn0_2_eq7}) with $x_{j}^{\left(3\right)} = 1$.\\
% ======











