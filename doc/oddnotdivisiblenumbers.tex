\chapter{Odd-Not-Divisible Numbers}
\label{ch:oddnotdivisiblenumbers}
\minitoc
%--------------------------------------------------
After we spent time with the set of all odd-divisible numbers, now, we switch to the set of all odd numbers which are not divisible by a particular other odd number.
%--------------------------------------------------
\section{Representation: Odd-Divisible Numbers}
\label{s:repodddivnumbers}
%--------------------------------------------------
Let's look again at (\ref{eq:oddnumbdiffpers1})
\[ y_{i,j}\left(x_{i},x_{j}\right) = 2\left(\left(2x_{i} + 1\right)x_{j} + x_{i}\right) + 1 ,\]
and its belonging values.
\begin{itemize}
	\item Be $x_{i} = 1$:
		\begin{equation}
			y_{1,j}\left(1,x_{j}\right) = 2\left(3x_{j} + 1\right) + 1, \quad x_{1,j} = 3x_{j} + 1
		\label{eq:repx1odddivnumb}
		\end{equation}

		\begin{table}[H]
		\centering
		\caption{The first ten values for (\ref{eq:repx1odddivnumb}).}
		\begin{tabular}{c|cccccccccc}
  			$x_{j}$ & 1 & 2 & 3 & 4 & 5 & 6 & 7 & 8 & 9 & 10 \\
  		\hline	$x_{1,j}$ & 4 & 7 & 10 & 13 & 16 & 19 & 22 & 25 & 28 & 31 \\
  			$y_{1,j}$ & 9 & 15 & 21 & 27 & 33 & 39 & 45 & 51 & 57 & 63 \\
		\end{tabular}
		\label{tab:repoddivnumbx1}
		\end{table}

	\item Be $x_{i} = 2$:
		\begin{equation}
			y_{2,j}\left(2,x_{j}\right) = 2\left(5x_{j} + 2\right) + 1, \quad x_{2,j} = 5x_{j} + 2
		\label{eq:repx2odddivnumb}
		\end{equation}

		\begin{table}[H]
		\centering
		\caption{The first ten values for (\ref{eq:repx2odddivnumb}).}
		\begin{tabular}{c|cccccccccc}
  			$x_{j}$ & 1 & 2 & 3 & 4 & 5 & 6 & 7 & 8 & 9 & 10 \\
  		\hline	$x_{2,j}$ & 7 & 12 & 17 & 23 & 28 & 33 & 38 & 43 & 48 & 53 \\
  			$y_{2,j}$ & 15 & 25 & 35 & 47 & 57 & 67 & 77 & 87 & 97 & 107 \\
		\end{tabular}
		\label{tab:repoddivnumbx2}
		\end{table}
	
	\item Be $x_{i} = 3$:
		\begin{equation}
			y_{3,j}\left(3,x_{j}\right) = 2\left(7x_{j} + 3\right) + 1, \quad x_{3,j} = 7x_{j} + 3
		\label{eq:repx3odddivnumb}
		\end{equation}

		\begin{table}[H]
		\centering
		\caption{The first ten values for (\ref{eq:repx3odddivnumb}).}
		\begin{tabular}{c|cccccccccc}
  			$x_{j}$ & 1 & 2 & 3 & 4 & 5 & 6 & 7 & 8 & 9 & 10 \\
  		\hline	$x_{3,j}$ & 10 & 17 & 24 & 31 & 38 & 45 & 52 & 59 & 66 & 73 \\
  			$y_{3,j}$ & 21 & 35 & 49 & 63 & 77 & 91 & 105 & 119 & 133 & 147 \\
		\end{tabular}
		\label{tab:repoddivnumbx3}
		\end{table}
	\item Be $x_{i} = \dots$: $\dots$.
\label{it:repodddivnumb}
\end{itemize}

%--------------------------------------------------
\section{Representation: Odd-Not-Divisible Numbers}
\label{s:repoddnotdivnumbers}
%--------------------------------------------------
Now, we take again $y_{i,j}\left(x_{i},x_{j}\right) = 2\left(\left(2x_{i} + 1\right)x_{j} + x_{i}\right) + 1$ and rephrase it into an equation which descripes all odd numbers which are not divisible by $2x_{i} + 1$.











