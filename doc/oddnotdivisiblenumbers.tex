\chapter{Odd-Not-Divisible Numbers}
\label{ch:oddnotdivisiblenumbers}
\minitoc
%--------------------------------------------------
After we spent time with the set of all odd-divisible numbers, now, we switch to the set of all odd numbers which are not divisible by a particular other odd number.
%--------------------------------------------------
\section{Representation: Odd-Divisible Numbers}
\label{s:repodddivnumbers}
%--------------------------------------------------
Let's look again at (\ref{eq:oddnumbdiffpers1})
\[ y_{i,j}\left(x_{i},x_{j}\right) = 2\left(\left(2x_{i} + 1\right)x_{j} + x_{i}\right) + 1 ,\]
and its belonging values.
\begin{itemize}
	\item Be $x_{1} = 1$:
		\begin{equation}
			y_{1,j}\left(1,x_{j}\right) = 2\left(3x_{j} + 1\right) + 1, \quad x_{1,j} = 3x_{j} + 1
		\label{eq:repx1odddivnumb}
		\end{equation}

		\begin{table}[H]
		\centering
		\caption{The first ten values for (\ref{eq:repx1odddivnumb}).}
		\begin{tabular}{c|cccccccccc}
  			$x_{j}$ & 1 & 2 & 3 & 4 & 5 & 6 & 7 & 8 & 9 & 10 \\
  		\hline	$x_{1,j}$ & 4 & 7 & 10 & 13 & 16 & 19 & 22 & 25 & 28 & 31 \\
  			$y_{1,j}$ & 9 & 15 & 21 & 27 & 33 & 39 & 45 & 51 & 57 & 63 \\
		\end{tabular}
		\label{tab:repoddivnumbx1}
		\end{table}

	\item Be $x_{2} = 2$:
		\begin{equation}
			y_{2,j}\left(2,x_{j}\right) = 2\left(5x_{j} + 2\right) + 1, \quad x_{2,j} = 5x_{j} + 2
		\label{eq:repx2odddivnumb}
		\end{equation}

		\begin{table}[H]
		\centering
		\caption{The first ten values for (\ref{eq:repx2odddivnumb}).}
		\begin{tabular}{c|cccccccccc}
  			$x_{j}$ & 1 & 2 & 3 & 4 & 5 & 6 & 7 & 8 & 9 & 10 \\
  		\hline	$x_{2,j}$ & 7 & 12 & 17 & 23 & 28 & 33 & 38 & 43 & 48 & 53 \\
  			$y_{2,j}$ & 15 & 25 & 35 & 47 & 57 & 67 & 77 & 87 & 97 & 107 \\
		\end{tabular}
		\label{tab:repoddivnumbx2}
		\end{table}
	
	\item Be $x_{3} = 3$:
		\begin{equation}
			y_{3,j}\left(3,x_{j}\right) = 2\left(7x_{j} + 3\right) + 1, \quad x_{3,j} = 7x_{j} + 3
		\label{eq:repx3odddivnumb}
		\end{equation}

		\begin{table}[H]
		\centering
		\caption{The first ten values for (\ref{eq:repx3odddivnumb}).}
		\begin{tabular}{c|cccccccccc}
  			$x_{j}$ & 1 & 2 & 3 & 4 & 5 & 6 & 7 & 8 & 9 & 10 \\
  		\hline	$x_{3,j}$ & 10 & 17 & 24 & 31 & 38 & 45 & 52 & 59 & 66 & 73 \\
  			$y_{3,j}$ & 21 & 35 & 49 & 63 & 77 & 91 & 105 & 119 & 133 & 147 \\
		\end{tabular}
		\label{tab:repoddivnumbx3}
		\end{table}
	\item Be $x_{i} = \dots$: $\dots$.
\label{it:repodddivnumb}
\end{itemize}

Now, let us also have a look at the extension to $\mathbb{Z}$. At first, we do the change $x_{i} \rightarrow -x_{i}$. 
\begin{equation}
	y_{i,j}\left(x_{i},x_{j}\right) = -\left(2\left(\left(2\left(x_{i} - 1\right) + 1\right)x_{j} + \left(x_{i} - 1\right)\right) + 1\right)
\label{eq:repodddivnumbers_negi}
\end{equation}
To have attention on this case will still play an role in the next sections. Now, we do the change $x_{j} \rightarrow -x_{j}$.
\begin{equation}
	y_{i,j}\left(x_{i},x_{j}\right) = -\left(2\left(\left(2x_{i} + 1\right)\left(x_{j} - 1\right) + x_{i}\right) + 1\right)
\label{eq:repodddivnumbers_negj}
\end{equation}
That's a simple case. We don't have to do anymore. 
%--------------------------------------------------
\section{Representation: Odd-Not-Divisible Numbers}
\label{s:repoddnotdivnumbers}
%--------------------------------------------------
Now, we take again $y_{i,j}\left(x_{i},x_{j}\right) = 2\left(\left(2x_{i} + 1\right)x_{j} + x_{i}\right) + 1$ and rephrase it into an equation which descripes all odd numbers which are not divisible by $2x_{i} + 1$. \\
That's not very hard. We can write
\begin{equation}
	y_{i,j}\left(x_{i},x_{j}\right) = 2\left(\left(2x_{i} + 1\right)x_{j} + x_{i} - \mu\left(x_{i}\right)\right) + 1,
\label{eq:repoddnotdivnumb}
\end{equation}
with
\begin{equation}
	\mu\left(x_{i}\right) = 1,\dots,2x_{i}, \quad \mu\left(x_{i}\right) \in \mathbb{N}.
\label{eq:defmu}
\end{equation}

Let's have a short look at the first values for $x_{i} = 1,2,3$.
\begin{itemize}
	\item Be $x_{1} = 1$:
		\begin{equation}
			y_{1,j}\left(1,x_{j}\right) = 2\left(3x_{j} + 1 - \mu\left(1\right)\right) + 1, \quad \mu\left(1\right) = 1,2, \quad x_{1,j} = 3x_{j} + 1
		\label{eq:repx1oddnotdivnumb}
		\end{equation}

		\begin{table}[H]
		\centering
		\caption{The first values for (\ref{eq:repx1oddnotdivnumb}).}
		\begin{tabular}{c|cccccccccc}
  			$x_{j}$ & 1 &  & 2 &  & 3 &  & 4 &  & 5 &  \\
			$\mu\left(1\right)$ & 1 & 2 & 1 & 2 & 1 & 2 & 1 & 2 & 1 & 2 \\
  		\hline	$x_{1,j}$ & 2 & 3 & 5 & 6 & 8 & 9 & 11 & 12 & 14 & 15 \\
  			$y_{1,j}$ & 5 & 7 & 11 & 13 & 17 & 19 & 23 & 25 & 29 & 31 \\
		\end{tabular}
		\label{tab:repodnotdivnumbx1}
		\end{table}

	\item Be $x_{2} = 2$:
		\begin{equation}
			y_{2,j}\left(2,x_{j}\right) = 2\left(5x_{j} + 2 - \mu\left(2\right)\right) + 1, \quad \mu\left(2\right) = 1,\dots,4 \quad x_{2,j} = 5x_{j} + 1
		\label{eq:repx2oddnotdivnumb}
		\end{equation}

		\begin{table}[H]
		\centering
		\caption{The first values for (\ref{eq:repx2oddnotdivnumb}).}
		\begin{tabular}{c|cccccccccc}
  			$x_{j}$ & 1 &  &  &  & 2 &  &  &  & 3 &  \\
			$\mu\left(1\right)$ & 1 & 2 & 3 & 4 & 1 & 2 & 3 & 4 & 1 & 2 \\
  		\hline	$x_{2,j}$ & 3 & 4 & 5 & 6 & 8 & 9 & 10 & 11 & 13 & 14 \\
  			$y_{2,j}$ & 7 & 9 & 11 & 13 & 17 & 19 & 21 & 23 & 27 & 29 \\
		\end{tabular}
		\label{tab:repodnotdivnumbx2}
		\end{table}

	\item Be $x_{3} = 3$:
		\begin{equation}
			y_{3,j}\left(3,x_{j}\right) = 2\left(7x_{j} + 1 - \mu\left(3\right)\right) + 1, \quad \mu\left(3\right) = 1,\dots,6 \quad x_{3,j} = 7x_{j} + 1
		\label{eq:repx3oddnotdivnumb}
		\end{equation}

		\begin{table}[H]
		\centering
		\caption{The first values for (\ref{eq:repx3oddnotdivnumb}).}
		\begin{tabular}{c|cccccccccc}
  			$x_{j}$ & 1 &  &  &  &  &  & 1 &  &  &  \\
			$\mu\left(1\right)$ & 1 & 2 & 3 & 4 & 5 & 6 & 1 & 2 & 3 & 4 \\
  		\hline	$x_{3,j}$ & 4 & 5 & 6 & 7 & 8 & 9 & 11 & 12 & 13 & 14 \\
  			$y_{3,j}$ & 9 & 11 & 13 & 15 & 17 & 19 & 23 & 25 & 17 & 29 \\
		\end{tabular}
		\label{tab:repodnotdivnumbx3}
		\end{table}

	\item Be $x_{i} = \dots$: $\dots$
\label{it:exampledndn}
\end{itemize}

\begin{remark}[Value set]
	You can see, the valid value set start not till $x_{i,j} = x_{i} + 1$.
\label{lab:notdef}
\end{remark}
%--------------------------------------------------
\section{Odd-Not-Divisible Numbers: Intersection}
\label{s:oddnotdivnuminter}
%--------------------------------------------------
Now we look at the intersection of two equations of the type (\ref{eq:repoddnotdivnumb}) with (\ref{eq:defmu}). Hence, we start with 
\begin{alignat}{3}
	y_{i,j}^{\left(1\right)}\left(x_{i}^{\left(1\right)},x_{j}^{\left(1\right)}\right) &= 2\left(\left(2x_{i}^{\left(1\right)} + 1\right)x_{j}^{\left(1\right)} + x_{i}^{\left(1\right)} - \mu\left(x_{i}^{\left(1\right)}\right)\right) + 1 \notag \\
	\mu\left(x_{i}^{\left(1\right)}\right) &= 1,\dots,2x_{i}^{\left(1\right)} \label{eq:eqsstartint1} \\
	\mathrm{and} \quad y_{i,j}^{\left(2\right)}\left(x_{i}^{\left(2\right)},x_{j}^{\left(2\right)}\right) &= 2\left(\left(2x_{i}^{\left(2\right)} + 1\right)x_{j}^{\left(2\right)} + x_{i}^{\left(2\right)} - \mu\left(x_{i}^{\left(2\right)}\right)\right) + 1 \notag \\
	\mu\left(x_{i}^{\left(2\right)}\right) &= 1,\dots,2x_{i}^{\left(2\right)}. \label{eq:eqsstartint2}
\end{alignat}

We do the intersection:
\begin{alignat}{3}
	0 &= \left(2x_{i}^{\left(1\right)} + 1\right)x_{j}^{\left(1\right)} - \left(2x_{i}^{\left(2\right)} + 1\right)x_{j}^{\left(2\right)} + x_{i}^{\left(1\right)} - x_{i}^{\left(2\right)} - \mu\left(x_{i}^{\left(1\right)}\right) + \mu\left(x_{i}^{\left(2\right)}\right) \label{eq:intersection1} \\
	 &= \left(2x_{i}^{\left(1\right)} + 1\right)\left(x_{j}^{\left(1\right)} - x_{j}^{\left(2\right)}\right) - 2\Delta x_{i}^{\left(1,2\right)} x_{j}^{\left(2\right)} - \Delta x_{i}^{\left(1,2\right)} - \mu\left(x_{i}^{\left(1\right)}\right) + \mu\left(x_{i}^{\left(2\right)}\right)\label{eq:intersection2} \\
	 &= \left(2x_{i}^{\left(1\right)} + 1\right)\left(x_{j}^{\left(1\right)} - x_{j}^{\left(2\right)}\right) - \left(2x_{j}^{\left(2\right)} + 1\right)\Delta x_{i}^{\left(1,2\right)} - \mu\left(x_{i}^{\left(1\right)}\right) + \mu\left(x_{i}^{\left(2\right)}\right) \label{eq:intersection3}
\end{alignat}

For the second one, we used $x_{i}^{\left(2\right)} = x_{i}^{\left(1\right)} + \Delta x_{i}^{\left(1,2\right)}$, $x_{i}^{\left(2\right)} > x_{i}^{\left(1\right)}$ and $\Delta x_{i}^{\left(1,2\right)} \in \mathbb{N}$.\\
To solve (\ref{eq:intersection1}) respectively (\ref{eq:intersection2}), we recognize that we have the boundary constraint, that $\left(2x_{i}^{\left(1\right)} + 1\right)$ and $\left(2x_{i}^{\left(2\right)} + 1\right)$ must not have any common factors.\\

Let's look at the case $\Delta x_{i}^{\left(1,2\right)} = 1$.
\begin{alignat}{3}
	0 &= \left(2x_{i}^{\left(1\right)} + 1\right)\left(x_{j}^{\left(1\right)} - x_{j}^{\left(2\right)}\right) - 2x_{j}^{\left(2\right)} - 1 - \mu\left(x_{i}^{\left(1\right)}\right) + \mu\left(x_{i}^{\left(2\right)}\right)\label{eq:intersection2_caseDeltax1} \\
	 &= \left(x_{j}^{\left(1\right)} - x_{j}^{\left(2\right)}\right) - \left(2x_{i}^{\left(1\right)} + 1\right)^{-1} \left(2x_{j}^{\left(2\right)} + 1 + \mu\left(x_{i}^{\left(1\right)} \right) - \mu\left(x_{i}^{\left(2\right)}\right)\right) \label{eq:intersection2_caseDeltax1_s1}
\end{alignat}

Now, let be 
\begin{equation}
	x_{j}^{\left(2\right)} = \left(1 + \mu\left(x_{i}^{\left(1\right)}\right) - \mu\left(x_{i}^{\left(2\right)}\right)\right)x_{i}^{\left(1\right)}.
\label{eq:solxj2delta1}
\end{equation}

\begin{alignat}{3}
	0 &= x_{j}^{\left(1\right)} - \left(1 + \mu\left(x_{i}^{\left(1\right)}\right) - \mu\left(x_{i}^{\left(2\right)}\right)\right)x_{i}^{\left(1\right)} - \left(1 + \mu\left(x_{i}^{\left(1\right)}\right) - \mu\left(x_{i}^{\left(2\right)}\right)\right) \label{eq:soldelta1_1} \\
	&= x_{j}^{\left(1\right)} - \left(1 + \mu\left(x_{i}^{\left(1\right)}\right) - \mu\left(x_{i}^{\left(2\right)}\right)\right)\left(x_{i}^{\left(1\right)} + 1\right) \label{eq:soldelta1_2}.
\end{alignat}

It follows
\begin{equation}
	x_{j}^{\left(1\right)} = \left(1 + \mu\left(x_{i}^{\left(1\right)}\right) - \mu\left(x_{i}^{\left(2\right)}\right)\right)\left(x_{i}^{\left(1\right)} + 1\right).
\label{eq:solxj1delta1}
\end{equation}

The equations (\ref{eq:solxj2delta1}) and (\ref{eq:solxj1delta1}) give us one particular solution. It's trivial to see, that we receive all solutions on $\mathbb{Z}$ for
\begin{alignat}{3}
	x_{j}^{\left(1\right)} = \left(2\left(x_{i}^{\left(1\right)} + 1\right) + 1\right)z^{\left(1,2\right)} &+ \left(1 + \mu\left(x_{i}^{\left(1\right)}\right) - \mu\left(x_{i}^{\left(2\right)}\right)\right)\left(x_{i}^{\left(1\right)} + 1\right) \label{eq:solxj1delta1_all} \\
	x_{j}^{\left(2\right)} = \left(2x_{i}^{\left(1\right)} + 1\right)z^{\left(1,2\right)} &+ \left(1 + \mu\left(x_{i}^{\left(1\right)}\right) - \mu\left(x_{i}^{\left(2\right)}\right)\right)x_{i}^{\left(1\right)} \label{eq:solxj2delta1_all},
\end{alignat}
with $z^{\left(1,2\right)} \in \mathbb{Z}$.

\vspace{0.3cm}
Now, we switch to the general case for $\Delta x_{i}^{\left(1,2\right)}$. We start again with
\begin{alignat}{3}
	0 &= \left(2x_{i}^{\left(1\right)} + 1\right)\left(x_{j}^{\left(1\right)} - x_{j}^{\left(2\right)}\right) - 2\Delta x_{i}^{\left(1,2\right)} x_{j}^{\left(2\right)} - \Delta x_{i}^{\left(1,2\right)} - \mu\left(x_{i}^{\left(1\right)}\right) + \mu\left(x_{i}^{\left(2\right)}\right) \notag \\
	& = \left(x_{j}^{\left(1\right)} - x_{j}^{\left(2\right)}\right) - \left(2x_{i}^{\left(1\right)} + 1\right)^{-1} \left(2\Delta x_{i}^{\left(1,2\right)}x_{j}^{\left(2\right)} + \Delta x_{i}^{\left(1,2\right)} + \mu\left(x_{i}^{\left(1\right)} \right) - \mu\left(x_{i}^{\left(2\right)}\right)\right). \label{eq:intersection2_caseDeltaxgen_s1}
\end{alignat}

We can see that our Ansatz from the case $\Delta x_{i}^{\left(1,2\right)} = 1$ not works anymore, since we find $\Delta x_{i}^{\left(1,2\right)}$ only in two of our four terms in the second part of equation (\ref{eq:intersection2_caseDeltaxgen_s1}), and hence it is not possible to factor out it.\\
But we will see, that it only seems to be not possible. In fact, it is possible!\\

Let's look at $\mu\left(x_{i}\right)$ for our considerations. We know our belonging equation
\[ x_{i,j} = \left(2x_{i} + 1\right)x_{j} + x_{i} - \mu\left(x_{i}\right), \quad \mu\left(x_{i}\right) = 1,\dots,2x_{i}, \quad \mu\left(x_{i}\right) \in \mathbb{N}. \]
Now, we will assume, we have $\mu\left(x_{i}\right) \rightarrow \Delta x_{i}^{\left(1,2\right)}\mu\left(x_{i}\right)$ instead. With this transition, we get two questions, which are important to be answered.

\begin{enumerate}
	\item For which values of $\Delta x_{i}^{\left(1,2\right)}$ we receive divisible solutions for $y_{i,j}\left(x_{i,j}\right)$ and hence, which values for $\Delta x_{i}^{\left(1,2\right)}$ are prohibited?\\

	We have the following two equations
	\begin{alignat}{3}
		x_{i,j}\prime &= \left(2x_{i} + 1\right)x_{j}\prime + x_{i} \label{eq:question1_div} \\
		x_{i,j} &= \left(2x_{i} + 1\right)x_{j} + x_{i} - \Delta x_{i}^{\left(1,2\right)}\mu\left(x_{i}\right) \label{eq:question1_notdiv}.
	\end{alignat}
	$x_{i,j}\prime$ gives us all divisible numbers and $x_{i,j}$ all not-divisible numbers for one particular multiplication table given by $x_{i}$. We make the intersection and receive
	\begin{alignat}{3}
		0 &= \left(2x_{i} + 1\right)\left(x_{j} - x_{j}\prime\right) - \Delta x_{i}^{\left(1,2\right)}\mu\left(x_{i}\right). \label{eq:question1_intersec}
	\end{alignat}

	Later, in our recursive calculation steps, we will see, that $\left(2x_{i} + 1\right)$ is always a prime. Since also $\mu\left(x_{i}\right) = 1, \dots, 2x_{i}$, it follows from this two conditions and the uniqueness of prime factorization, that (\ref{eq:question1_intersec}) is only be fulfilled for $\Delta x_{i}^{\left(1,2\right)} = n\left(2x_{i} + 1\right)$, $n \in \mathbb{Z}$.\\

	Thankfully, this case can never happen during our recursive prime number calculation! If we would choose $\Delta x_{i}^{\left(1,2\right)} = n\left(2x_{i} + 1\right)$, we would land on our divisible numbers for this multiplication table, and that is exactly NOT, what we want. Hence, we don't have this case.

	\item Is the result of the transition $\mu\left(x_{i}\right) \rightarrow \Delta x_{i}^{\left(1,2\right)}\mu\left(x_{i}\right)$ still surjective?\\

	We take
	\begin{alignat}{3}
		x_{i,j} &= \left(2x_{i} + 1\right)x_{j} + x_{i} - \mu\left(x_{i}\right) \label{eq:question2_eq1} \\
		x_{i,j}\prime &= \left(2x_{i} + 1\right)x_{j}\prime + x_{i} - \Delta x_{i}^{\left(1,2\right)}\mu\prime\left(x_{i}\right) \label{eq:question2_eq2}.
	\end{alignat}
	$x_{i,j}$ is the equation before and $x_{i,j}\prime$ after the transition. For the intersection, we receive
	\begin{alignat}{3}
		0 &= \underbrace{\left(2x_{i} + 1\right)\left(x_{j} - x_{j}\prime\right)}_{*} - \underbrace{\mu\left(x_{i}\right)}_{**} + \underbrace{\Delta x_{i}^{\left(1,2\right)}\mu\prime\left(x_{i}\right)}_{***} \label{eq:question2_intersec} \\
		\Leftrightarrow \quad \mu\left(x_{i}\right) &= \left(2x_{i} + 1\right)\left(x_{j} - x_{j}\prime\right) + \Delta x_{i}^{\left(1,2\right)}\mu\prime\left(x_{i}\right). \label{eq:question2_intersec}
	\end{alignat}
	
	We know $\left(2x_{i} + 1\right)$ is prime, $\mu\left(x_{i}\right), \mu\prime\left(x_{i}\right) \in \{1, \cdots, 2x_{i}\}$ and $\Delta x_{i}^{\left(1,2\right)} \neq 2x_{i} + 1$. From $*$ follows that $** + ***$ has to be an integer multiple from $2x_{i} + 1$. Since neither $\Delta x_{i}^{\left(1,2\right)}$ nor $\mu\prime\left(x_{i}\right)$ can be an integer multiple of $2x_{i} + 1$, we know $***$ is also never an integer multiple of $2x_{i} + 1$. But also, caused by $*$ we know the maximum gap between two generated numbers can only be $2x_{i}$. Caused by the definition range of $**$ it follows, the trueness of the given assumption, that the result of the transition is still surjective.

\label{en:transitionquestions}
\end{enumerate}

After getting our answers to this two questions, let's go back to our original problem.\\

The results of our questions show that it isn't a problem, to apply the transition $\mu\left(x_{i}\right) \rightarrow \Delta x_{i}^{\left(1,2\right)}\mu\left(x_{i}\right)$ on our equation (\ref{eq:intersection2_caseDeltaxgen_s1})
\begin{alignat}{3}
	0 & = \left(x_{j}^{\left(1\right)} - x_{j}^{\left(2\right)}\right) - \left(2x_{i}^{\left(1\right)} + 1\right)^{-1} \Delta x_{i}^{\left(1,2\right)}\left(2x_{j}^{\left(2\right)} + 1 + \mu\left(x_{i}^{\left(1\right)} \right) - \mu\left(x_{i}^{\left(2\right)}\right)\right). \label{eq:intersection2_caseDeltaxgen_s2_transition_s1}
\end{alignat}

Because we know that $\Delta x_{i}^{\left(1,2\right)} \neq 2x_{i}^{\left(1\right)} + 1$, we can do this without concerns. If we compare (\ref{eq:intersection2_caseDeltaxgen_s2_transition_s1}) with (\ref{eq:intersection2_caseDeltax1_s1}) from our $\Delta x_{i}^{\left(1,2\right)} = 1$ case, we see that our new equation differs from the old one, only by the factor $\Delta x_{i}^{\left(1,2\right)}$ in front of the second term.\\

Now, let be again
\begin{equation}
	x_{j}^{\left(2\right)} = \left(1 + \mu\left(x_{i}^{\left(1\right)}\right) - \mu\left(x_{i}^{\left(2\right)}\right)\right)x_{i}^{\left(1\right)}.
\label{eq:solxj2deltagen}
\end{equation}

After analog calculation of our prior case, we receive
\begin{equation}
	x_{j}^{\left(1\right)} = \left(1 + \mu\left(x_{i}^{\left(1\right)}\right) - \mu\left(x_{i}^{\left(2\right)}\right)\right)\left(x_{i}^{\left(1\right)} + \Delta x_{i}^{\left(1,2\right)}\right).
\label{eq:solxj1deltagen}
\end{equation}

For all solutions on $\mathbb{Z}$ we have
\begin{alignat}{3}
	x_{j}^{\left(1\right)} = \left(2\left(x_{i}^{\left(1\right)} + \Delta x_{i}^{\left(1,2\right)}\right) + 1\right)z^{\left(1,2\right)} &+ \left(1 + \mu\left(x_{i}^{\left(1\right)}\right) - \mu\left(x_{i}^{\left(2\right)}\right)\right)\left(x_{i}^{\left(1\right)} + \Delta x_{i}^{\left(1,2\right)}\right) \label{eq:solxj1deltagen_all} \\
	x_{j}^{\left(2\right)} = \left(2x_{i}^{\left(1\right)} + 1\right)z^{\left(1,2\right)} &+ \left(1 + \mu\left(x_{i}^{\left(1\right)}\right) - \mu\left(x_{i}^{\left(2\right)}\right)\right)x_{i}^{\left(1\right)} \label{eq:solxj2deltagen_all},
\end{alignat}
with $z^{\left(1,2\right)} \in \mathbb{Z}$.\\

But have attention, that this is the solution for 
\begin{alignat}{3}
	x_{i,j}^{\left(1\right)}\left(x_{i}^{\left(1\right)}, x_{j}^{\left(1\right)}\right) &= \left(2x_{i}^{\left(1\right)} + 1\right)x_{j}^{\left(1\right)} + x_{i}^{\left(1\right)} - \Delta x_{i}^{\left(1,2\right)}\mu\left(x_{i}^{\left(1\right)}\right) \label{eq:xijeq_new1} \\
	x_{i,j}^{\left(2\right)}\left(x_{i}^{\left(2\right)}, x_{j}^{\left(2\right)}\right) &= \left(2x_{i}^{\left(2\right)} + 1\right)x_{j}^{\left(2\right)} + x_{i}^{\left(2\right)} - \Delta x_{i}^{\left(1,2\right)}\mu\left(x_{i}^{\left(2\right)}\right) \label{eq:xijeq_new2}
\end{alignat}
now! $\mu\left(x_{i}^{\left(1\right)}\right)$ and $\mu\left(x_{i}^{\left(2\right)}\right)$ are the same like before!
%--------------------------------------------------
\section{Intersection of an arbitrary number of equations}
\label{s:intarbitrarynumbeqs}
%--------------------------------------------------
Until now, we only had the intersection of two equations. Now we will switch to the case of an arbitrary number of equations. So, now, we have
\begin{alignat}{3}
	x_{i,j}^{\left(k\right)}\left(x_{i}^{\left(k\right)}, x_{j}^{\left(k\right)}\right) &= \left(2x_{i}^{\left(k\right)} + 1\right)x_{j}^{\left(k\right)} + x_{i}^{\left(k\right)} - \Delta x_{i}^{\left(k,k\prime\right)}\mu\left(x_{i}^{\left(k\right)}\right), \label{eq:intarbnumb_geneq}
\end{alignat}
with $k \in \mathbb{N}$, $k \neq k\prime$. Ok. What is the general solution of the intersection of $n$ equations? Let's start again with two equations. We receive
\begin{alignat}{3}
	x_{j}^{\left(1\right)} &= \left(2x_{i}^{\left(2\right)} + 1\right)z^{\left(1,2\right)} + \left(1 + \mu\left(x_{i}^{\left(1\right)}\right) - \mu\left(x_{i}^{\left(2\right)}\right)\right)x_{i}^{\left(2\right)} \label{eq:step1_sol_xj1} \\
	x_{j}^{\left(2\right)} &= \left(2x_{i}^{\left(1\right)} + 1\right)z^{\left(1,2\right)} + \left(1 + \mu\left(x_{i}^{\left(1\right)}\right) - \mu\left(x_{i}^{\left(2\right)}\right)\right)x_{i}^{\left(1\right)} \label{eq:step1_sol_xj2}
\end{alignat}
and finally
\begin{alignat}{3}
	x_{i,j}^{\left(1,2\right)} =& \, x_{i,j}^{\left(1\right)}\left(x_{i}^{\left(1\right)},x_{j}^{\left(1\right)}\right) \notag \\
	=& \ \left(2x_{i}^{\left(1\right)} + 1\right)\left(2x_{i}^{\left(2\right)} + 1\right)z^{\left(1,2\right)}
	+ \left(2x_{i}^{\left(1\right)} + 1\right)\left(1 + \mu\left(x_{i}^{\left(1\right)}\right) - \mu\left(x_{i}^{\left(2\right)}\right)\right)x_{i}^{\left(2\right)} \notag \\  
	&+ x_{i}^{\left(1\right)} - \left(x_{i}^{\left(2\right)} - x_{i}^{\left(1\right)}\right)\mu\left(x_{i}^{\left(1\right)}\right) \notag \\
	=& \ \underbrace{\left(2x_{i}^{\left(1\right)} + 1\right)\left(2x_{i}^{\left(2\right)} + 1\right)}_{=: \, 2x_{i}^{\left(1,2\right)} + 1}z^{\left(1,2\right)}
	+ \underbrace{2x_{i}^{\left(1\right)}x_{i}^{\left(2\right)} + x_{i}^{\left(1\right)} + x_{i}^{\left(2\right)}}_{=: \, x_{i}^{\left(1,2\right)}} \notag \\
	&- \left(\underbrace{-\mu\left(x_{i}^{\left(1\right)}\right)x_{i}^{\left(1\right)}\left(2x_{i}^{\left(2\right)} + 1\right)
	+ \mu\left(x_{i}^{\left(2\right)}\right)x_{i}^{\left(2\right)}\left(2x_{i}^{\left(1\right)} + 1\right)}_{=: \, \mu\left(x_{i}^{\left(1,2\right)}\right)}\right) \notag \\
	=& \ \left(2x_{i}^{\left(1,2\right)} + 1\right)z^{\left(1,2\right)} + x_{i}^{\left(1,2\right)} - \mu\left(x_{i}^{\left(1,2\right)}\right). \label{eq:step1_sol_xij12}
\end{alignat}

In (\ref{eq:step1_sol_xij12}), we make the transition $\mu\left(x_{i}^{\left(1,2\right)}\right) \rightarrow \Delta x_{i}^{\left(\left(1,2\right),3\right)}\mu\left(x_{i}^{\left(1,2\right)}\right)$ for the intersection with the third equation. We have
\begin{alignat}{3}
	z^{\left(1,2\right)} &= \left(2x_{i}^{\left(3\right)} + 1\right)z^{\left(\left(1,2\right),3\right)} + \left(1 + \mu\left(x_{i}^{\left(1,2\right)}\right) - \mu\left(x_{i}^{\left(3\right)}\right)\right)x_{i}^{\left(3\right)} \label{eq:step1_sol_z12} \\
	x_{j}^{\left(3\right)} &= \left(2x_{i}^{\left(1,2\right)} + 1\right)z^{\left(\left(1,2\right),3\right)} + \left(1 + \mu\left(x_{i}^{\left(1,2\right)}\right) - \mu\left(x_{i}^{\left(3\right)}\right)\right)x_{i}^{\left(1,2\right)} \label{eq:step1_sol_xj3}
\end{alignat}
and finally
\begin{alignat}{3}
	x_{i,j}^{\left(\left(1,2\right),3\right)} =& \, x_{i,j}^{\left(1,2\right)}\left(x_{i}^{\left(1,2\right)},z^{\left(1,2\right)}\right) \notag \\
	=& \ \left(2x_{i}^{\left(1,2\right)} + 1\right)\left(2x_{i}^{\left(3\right)} + 1\right)z^{\left(\left(1,2\right),3\right)}
	+ \left(2x_{i}^{\left(1,2\right)} + 1\right)\left(1 + \mu\left(x_{i}^{\left(1,2\right)}\right) - \mu\left(x_{i}^{\left(3\right)}\right)\right)x_{i}^{\left(3\right)} \notag \\  
	&+ x_{i}^{\left(1,2\right)} - \left(x_{i}^{\left(3\right)} - x_{i}^{\left(1,2\right)}\right)\mu\left(x_{i}^{\left(1,2\right)}\right) \notag \\
	=& \ \left(2x_{i}^{\left(1\right)} + 1\right)\left(2x_{i}^{\left(2\right)} + 1\right)\left(2x_{i}^{\left(3\right)} + 1\right)z^{\left(\left(1,2\right),3\right)}
	+ 2x_{i}^{\left(1,2\right)}x_{i}^{\left(3\right)} + x_{i}^{\left(1,2\right)} + x_{i}^{\left(3\right)} \notag \\
	&- \left(-\mu\left(x_{i}^{\left(1,2\right)}\right)x_{i}^{\left(1,2\right)}\left(2x_{i}^{\left(3\right)} + 1\right)
	+ \mu\left(x_{i}^{\left(3\right)}\right)x_{i}^{\left(3\right)}\left(2x_{i}^{\left(1,2\right)} + 1\right)\right) \notag \\
	=& \ \left(2x_{i}^{\left(1\right)} + 1\right)\left(2x_{i}^{\left(2\right)} + 1\right)\left(2x_{i}^{\left(3\right)} + 1\right)z^{\left(\left(1,2\right),3\right)} \notag \\
	&+ \ 2^{2}x_{i}^{\left(1\right)}x_{i}^{\left(2\right)}x_{i}^{\left(3\right)} + 2^{1}\left(x_{i}^{\left(1\right)}x_{i}^{\left(3\right)} + x_{i}^{\left(2\right)}x_{i}^{\left(3\right)} + x_{i}^{\left(1\right)}x_{i}^{\left(2\right)}\right) + 2^{0}\left(x_{i}^{\left(1\right)} + x_{i}^{\left(2\right)} + x_{i}^{\left(3\right)}\right) \notag \\
	&- \left(\mu\left(x_{i}^{\left(1\right)}\right)x_{i}^{\left(1\right)}\left(2x_{i}^{\left(2\right)} + 1\right)
	- \mu\left(x_{i}^{\left(2\right)}\right)x_{i}^{\left(2\right)}\left(2x_{i}^{\left(1\right)} + 1\right)\right) \notag \\
	&\cdot \left(2x_{i}^{\left(1\right)}x_{i}^{\left(2\right)} + x_{i}^{\left(1\right)} + x_{i}^{\left(2\right)}\right)\left(2x_{i}^{\left(3\right)} + 1\right) \notag \\
	&- \left(\mu\left(x_{i}^{\left(3\right)}\right)x_{i}^{\left(3\right)}\left(2x_{i}^{\left(1\right)} + 1\right)\left(2x_{i}^{\left(2\right)} + 1\right)\right) \notag \\
	=& \ \left(2x_{i}^{\left(1\right)} + 1\right)\left(2x_{i}^{\left(2\right)} + 1\right)\left(2x_{i}^{\left(3\right)} + 1\right)z^{\left(\left(1,2\right),3\right)} \notag \\
	&+ \frac{1}{2}\left(\left(2x_{i}^{\left(1\right)} + 1\right)\left(2x_{i}^{\left(2\right)} + 1\right)\left(2x_{i}^{\left(3\right)} + 1\right) - 1\right) \notag \\
	&- \left(\mu\left(x_{i}^{\left(1\right)}\right) x_{i}^{\left(1\right)}\left(2x_{i}^{\left(2\right)} + 1\right)\frac{1}{2}\left(\left(2x_{i}^{\left(1\right)} + 1\right)\left(2x_{i}^{\left(2\right)} + 1\right) - 1\right)\left(2x_{i}^{\left(3\right)} + 1\right)\right) \notag \\
	&- \left(-\mu\left(x_{i}^{\left(2\right)}\right) x_{i}^{\left(2\right)}\left(2x_{i}^{\left(1\right)} + 1\right)\frac{1}{2}\left(\left(2x_{i}^{\left(1\right)} + 1\right)\left(2x_{i}^{\left(2\right)} + 1\right) - 1\right)\left(2x_{i}^{\left(3\right)} + 1\right)\right) \notag \\
	&- \left(\mu\left(x_{i}^{\left(3\right)}\right)x_{i}^{\left(3\right)}\left(2x_{i}^{\left(1\right)} + 1\right)\left(2x_{i}^{\left(2\right)} + 1\right)\right). \label{eq:step1_sol_xij123}
\end{alignat}

For the case of the intersection of $n$ ($n > 1$) equations, we can write
\begin{alignat}{3}
	x_{i,j}^{\left(1 \dots n\right)} =& \ \prod_{k=1}^{n}\left(2x_{i}^{\left(k\right)} + 1\right)z^{\left(1 \dots n\right)} \notag \\
	&+ \frac{1}{2} \left(\prod_{k=1}^{n} \left(2x_{i}^{\left(k\right)} + 1\right) - 1\right) \notag \\
	&- \left(\left(-1\right)^{n+1}\mu\left(x_{i}^{\left(1\right)}\right)x_{i}^{\left(1\right)} \prod_{l\neq 1}^{n} \left(2x_{i}^{\left(l\right)} + 1\right) \prod_{f=1}^{n-2, n > 2} \left(\frac{1}{2}\left(\prod_{m=1}^{f+1} \left(2x_{i}^{\left(m\right)} + 1\right) - 1\right)\right) \right) \notag \\
	&- \left(\sum_{k=2}^{n} \left(-1\right)^{n+k}\mu\left(x_{i}^{\left(k\right)}\right)x_{i}^{\left(k\right)} \prod_{l\neq k}^{n} \left(2x_{i}^{\left(l\right)} + 1\right) \prod_{f=1}^{n-k, n > 2} \left(\frac{1}{2} \left(\prod_{m=1}^{f+1} \left(2x_{i}^{\left(m\right)} + 1\right) - 1\right)\right)\right). \label{eq:gensol_neqs}
\end{alignat}
Here we used the definition
\begin{equation}
	\prod_{f=1}^{0} A_{f} := 1.
\label{eq:defprod}
\end{equation}

Like in the section before, we have to take a look at the $\Delta x_{i}^{\left(\left(1,2\right),\, \dots\right)}$ values from our transition. Now, for the intersection of an arbitrary number of equations, we have the case of pre-factors of $z$ which are the product of the already calculated primes. Given be
\begin{alignat}{3}
	x_{i,j}^{\left(1 \dots n\right)} &= \left(2x_{i}^{\left(1\right)} + 1\right)\left(2x_{i}^{\left(2 \dots n\right)} + 1\right)z^{\left(1 \dots n\right)} + x_{i}^{\left(1, \dots n\right)} - \Delta x_{i}^{\left(\left(1 \dots n\right),n+1\right)}\mu\left(x_{i}^{\left(1 \dots n\right)}\right) \label{eq:arbnumeqdeltaxtrans}
\end{alignat}
with
\begin{equation}
	\Delta x_{i}^{\left(\left(1 \dots n\right),n + 1\right)} = \left(2x_{i}^{\left(1\right)} + 1\right) \quad \mathrm{and} \quad \mu\left(x_{i}^{\left(1 \dots n\right)}\right) = 1, \dots, 2x_{i}^{\left(1 \dots n\right)}.
\label{eq:deltaalphawcomfac}
\end{equation}
It's easy to see that this could lead to problems, since now, also the case $\mu\left(x_{i}^{1 \dots n}\right) = \left(2x_{i}^{\left(2 \dots n\right)} + 1\right)$ is possible and we would receive again the equation for divisible, instead of not-divisible numbers again.\\

But thankfully, during our recursion for prime number generation, we will never have this case!\\

Assume we would have given (\ref{eq:deltaalphawcomfac})
\begin{alignat}{3}
	x_{i,j}^{\left(1 \dots n\right)} =& \ \left(2x_{i}^{\left(1\right)} + 1\right)\left(2x_{i}^{\left(2 \dots n\right)} + 1\right)z^{\left(1 \dots n\right)} \notag \\
	&+ \left(\left(2x_{i}^{\left(1\right)} + 1\right)\left(2x_{i}^{\left(2 \dots n\right)} + 1\right) - 1\right)\frac{1}{2} \notag \\
	& - \left(2x_{i}^{\left(1\right)} + 1\right)\mu\left(x_{i}^{\left(1 \dots n\right)}\right). \label{eq:deltafactorsense1}
\end{alignat}

Now, assume we would have $x_{i}^{\left(n+1\right)} = x_{i}^{\left(1 \dots n\right)} \pm \left(2x_{i}^{\left(1\right)} + 1\right)$. With this, we would receive
\begin{alignat}{3}
	x_{i,j}^{\left(n+1\right)} =& \ \left(2\left(\left(\left(2x_{i}^{\left(1\right)} + 1\right)\left(2x_{i}^{\left(2 \dots n\right)} + 1\right) - 1\right)\frac{1}{2} \pm \left(2x^{\left(1\right)}_{i} + 1\right)\right) + 1\right)x_{j}^{\left(n+1\right)} \notag \\
	&+ \left(\left(2x_{i}^{\left(1\right)} + 1\right)\left(2x_{i}^{\left(2 \dots n\right)} + 1\right) - 1\right)\frac{1}{2} \pm \left(2x_{i}^{\left(1\right)} + 1\right)\notag \\
	&\pm \left(2x_{i}^{\left(1\right)} + 1\right)\mu\left(x_{i}^{\left(1 \dots n\right)} \pm \left(2x_{i}^{\left(1\right)} + 1\right)\right) \notag \\
	=& \ \left(2x_{i}^{\left(1\right)} + 1\right)\left(\left(2x_{i}^{\left(2 \dots n\right)} + 1\right) \pm 2\right)x_{j}^{\left(n+1\right)} \notag \\
	&+ \left(2x_{i}^{\left(1\right)} + 1\right)\left(\left(2x_{i}^{\left(2 \dots n\right)} + 1\right)\frac{1}{2} \pm 1\right) - \frac{1}{2} \notag \\
	&\pm \left(2x_{i}^{\left(1\right)} + 1\right)\mu\left(x_{i}^{\left(1 \dots n\right)} \pm \left(2x_{i}^{\left(1\right)} + 1\right)\right) \notag \\
	=& \ \left(2x_{i}^{\left(1\right)} + 1\right)\left(\left(2x_{i}^{\left(2 \dots n\right)} + 1\right) \pm 2\right)x_{j}^{\left(n+1\right)} \notag \\
	&+ \ x_{i}^{\left(1\right)} + \left(2x_{i}^{\left(1\right)} + 1\right)\left(x^{\left(2 \dots n\right)} \pm 1\right) \notag \\ 
	&\pm \left(2x_{i}^{\left(1\right)} + 1\right)\mu\left(x_{i}^{\left(1 \dots n\right)} \pm \left(2x_{i}^{\left(1\right)} + 1\right)\right). \label{eq:deltafactorsense2}
\end{alignat}
We can see that we get an equation for a multiplication table for which we have already done our recursion before. Since, the case $\Delta x_{i}^{\left(\left(1 \dots n\right),n + 1\right)} = \left(2x_{i}^{\left(1\right)} + 1\right)$ would make no sense for our recursion.
\begin{comment} % This section is useless
%--------------------------------------------------
\section{Modification: Shifting of $x_{j}$}
\label{s:modshiftingxj}
%--------------------------------------------------
We will repeat the steps of the last two section but with a small modification, now. We will introduce a shifting of $x_{j}$. So at first, let's look at
\begin{alignat}{3}
	x_{i,j}^{\left(k\right)}\left(x_{i}^{\left(k\right)}, x_{j}^{\left(k\right)}\right) &= \left(2x_{i}^{\left(k\right)} + 1\right)x_{j}^{\left(k\right)} + x_{i}^{\left(k\right)} - \Delta x_{i}^{\left(k,k\prime\right)}\mu\left(x_{i}^{\left(k\right)}\right) + \left(2x_{i}^{\left(k\right)} + 1\right)s_{j}^{\left(k\right)} \notag \\
	\Leftrightarrow \quad &= \left(2x_{i}^{\left(k\right)} + 1\right)\left(\underbrace{x_{j}^{\left(k\right)} + s_{j}^{\left(k\right)}}_{=: \, x_{j,s}^{\left(k\right)}}\right) + x_{i}^{\left(k\right)} - \Delta x_{i}^{\left(k,k\prime\right)}\mu\left(x_{i}^{\left(k\right)}\right), \label{eq:intarbnumb_geneq_shift}
\end{alignat}
with $s_{j}^{\left(k\right)} \in \mathbb{Z}$. That's a small, but important modification, of equation (\ref{eq:intarbnumb_geneq}). This modification has a nice influence on our intersection and gives us new possibilities for finding a final solution. We have again two equations
\begin{alignat}{3}
	x_{i,j}^{\left(1\right)}\left(x_{i}^{\left(1\right)}, x_{j}^{\left(1\right)}\right) &= \left(2x_{i}^{\left(1\right)} + 1\right)\left(x_{j}^{\left(1\right)} + s_{j}^{\left(1\right)}\right) + x_{i}^{\left(1\right)} - \Delta x_{i}^{\left(1,2\right)}\mu\left(x_{i}^{\left(1\right)}\right) \label{eq:eq1_shifted} \\
	x_{i,j}^{\left(2\right)}\left(x_{i}^{\left(2\right)}, x_{j}^{\left(2\right)}\right) &= \left(2x_{i}^{\left(2\right)} + 1\right)\left(x_{j}^{\left(2\right)} + s_{j}^{\left(2\right)}\right) + x_{i}^{\left(2\right)} - \Delta x_{i}^{\left(1,2\right)}\mu\left(x_{i}^{\left(2\right)}\right) \label{eq:eq2_shifted}
\end{alignat}
and the solution of their intersection.
\begin{alignat}{3}
	x_{j,s}^{\left(1\right)} &:= x_{j}^{\left(1\right)} + s_{j}^{\left(1\right)} &= \left(2x_{i}^{\left(2\right)} + 1\right)z^{\left(1,2\right)} + \left(1 + \mu\left(x_{i}^{\left(1\right)}\right) - \mu\left(x_{i}^{\left(2\right)}\right)\right)x_{i}^{\left(2\right)} \label{eq:step1_sol_xj1_shifted} \\
	x_{j,s}^{\left(2\right)} &:= x_{j}^{\left(2\right)} + s_{j}^{\left(2\right)} &= \left(2x_{i}^{\left(1\right)} + 1\right)z^{\left(1,2\right)} + \left(1 + \mu\left(x_{i}^{\left(1\right)}\right) - \mu\left(x_{i}^{\left(2\right)}\right)\right)x_{i}^{\left(1\right)}. \label{eq:step1_sol_xj2_shifted}
\end{alignat}
We know, it can become difficult to see the $z^{\left(1,2\right)}$-values for receiving the valid $x_{j}$-range for our recursion step. With this new parameter $s_{j}$, we have the possibility to modify the right part of our soultion equations in such a way, which makes it easier to find our valid values, now. Our $x_{i,j}$ is of course the same like before.
\end{comment}
